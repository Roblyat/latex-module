\chapter{Summary and Outlook}

\section{Discussion and Outlook}
\label{sec:discussion_outlook}

    The results in this thesis can be interpreted as a first step towards a more
    general notion of \emph{dynamic awareness} for collaborative manipulators.
    Several conceptual implications and natural extensions are worth highlighting.

    \paragraph{Online torque awareness for safety–critical control.}
    Once the DeLaN+LSTM architecture has been trained and deployed, the robot
    possesses an accurate estimate of its joint torques
    $\hat{\boldsymbol{\tau}}_{\mathrm{RG},k}$ at every time step, and---via the
    Jacobian---of the corresponding end–effector wrench
    $\hat{\vec{F}}_{\mathrm{RG},k}$ in the measurement frame. In principle, this
    enables model–predictive controllers and safety monitors to operate directly on
    end–effector torque or wrench limits, for example those prescribed for
    collaborative operation, rather than relying on conservative joint–space
    bounds. By exploiting the available torque margin more tightly, such a
    controller could realise higher velocity and acceleration profiles while
    remaining within certified interaction–force constraints.

    \paragraph{Beyond joint space: evaluation of end–effector wrench prediction.}
    The present thesis restricts training and quantitative evaluation to joint
    space: the DeLaN+LSTM model is trained on joint states and motor torques, and
    its quality is assessed primarily via joint–torque prediction errors. A
    natural next step is a systematic evaluation of end–effector wrench prediction
    against ground–truth FT data at the flange. One promising route is to combine
    IsaacSim–based data generation (with and without domain randomisation and
    sensor noise) with real–robot experiments using the same excitation families
    as in this work. This would allow a comprehensive comparison between
    \emph{(i)} models trained purely in simulation and tested on the real robot,
    \emph{(ii)} models trained on real data and tested on held–out real
    trajectories, and \emph{(iii)} hybrid training schemes that mix simulated and
    measured data. Such an evaluation pipeline would provide strong evidence for
    the robustness of the joint–space training strategy when viewed in the
    measurement frame. Once this consistency between Jacobian–mapped model predictions 
    and ground–truth flange wrenches has been demonstrated, the same DeLaN+LSTM architecture 
    can, in principle, be trained and deployed using proprioceptive data alone, without
    requiring an FT sensor in the loop.

    \paragraph{Tool/gripper calibration as a third learning stage.}
    In the current experiments, the LSTM residual model in Stage~2 reduces the
    joint–torque error on the nominal robot–only configuration to near zero. When
    a fixed gripper is mounted, one could either re–record the entire dataset with
    the gripper attached or, more interestingly, treat the gripper as an additional
    structured disturbance to be learned. Conceptually, this suggests a
    third learning stage in which a small network (e.g.\ another LSTM) takes the
    residual error of Stage~2 as input and learns a tool–specific compensation
    term. From a closed–loop viewpoint, the robot would then iteratively refine
    its internal model when a new tool is attached, realising a form of transfer
    learning for tool calibration without rebuilding the entire inverse–dynamics
    model from scratch.

    \paragraph{Towards payload identification from residual wrenches.}
    Once tool/gripper effects are compensated, remaining discrepancies between the
    nominal robot–gripper wrench model and the measured flange wrench can be
    attributed primarily to external interaction and payload dynamics. Mapping the
    Stage~2/Stage~3 residuals into the end–effector frame via the Jacobian yields
    an estimate of the interaction wrench along and around the $x/y/z$ axes. In
    combination with the corresponding end–effector linear and angular velocities
    and accelerations, this opens the door to a further stage in which a dedicated
    estimator recovers the payload dynamic parameters
    $\boldsymbol{\phi}_{\mathrm{payload}} = (m,\mathbf{c},\mathbf{I}) \in \mathbb{R}^{10}$
    from these residual wrenches in a Newton–Euler framework. In other words, the
    same DeLaN+LSTM backbone that currently provides accurate joint–space
    prediction could serve, after appropriate extensions, as the core of an
    online PDPI pipeline that incrementally refines the effective rigid–body model
    (robot + tool + payload) during everyday manipulation tasks.

    \paragraph{Interpretable sparse models via SINDy.}
    The DeLaN+LSTM architecture used in this thesis is deliberately expressive but remains a
    largely black–box regression model: the Lagrangian network and the recurrent residual model
    jointly fit the data, yet they do not expose which physical effects are actually dominant in the
    identified dynamics. A complementary avenue for future work is to apply sparse identification
    of nonlinear dynamics (SINDy)~\cite{SOA_SINDy} to the same joint–space data. In such a
    setting, the candidate function library could explicitly include standard rigid–body torque
    models (e.g.\ Newton–Euler and Lagrangian terms), nonlinear Coriolis and gravity terms, and
    parametric friction models, together with a small number of additional generic nonlinear
    features. SINDy would then select a parsimonious subset of these terms that best explains the
    observed motor torques, potentially yielding an interpretable model in which a few physically
    meaningful contributions account for most of the variance in the data. This would provide
    insight into which dynamic effects the robot actually relies on in practice, and would offer a
    transparent counterpart to the high–capacity DeLaN+LSTM model used in the present work.%
