\chapter{Summary and Outlook}

This thesis addressed the problem of learning accurate inverse-dynamics models for collaborative manipulators in a way that remains compatible with the requirements of online payload and interaction-force estimation.
Rather than treating inverse dynamics as a purely black-box regression task, the work adopted a physics-inspired learning strategy that embeds mechanical structure into the model class and learns the remaining discrepancy from data.
Since the IEEE DataPort dataset used for evaluation provides motor-current measurements (but no joint-torque ground truth), Chapters~4--7 consistently formulated objectives and metrics in the motor-current domain, noting that torque-domain quantities can be recovered up to a constant motor torque constant in the operating regime considered.

The proposed approach is a two-stage pipeline.
Stage~1 uses Deep Lagrangian Networks (DeLaN) to learn a structured dynamics predictor from joint kinematics, yielding a baseline motor-current model that respects fundamental properties of rigid-body dynamics.
Stage~2 augments this baseline with a residual Long Short-Term Memory (LSTM) sequence model trained on the remaining modelling error, enabling history-dependent corrections that capture effects not explained by the instantaneous state alone.
To make both stages reproducible and comparable across dataset splits, the thesis introduced a trajectory-level evaluation protocol, an explicit best-model selection procedure that accounts for accuracy and seed stability, and an implementation that runs end-to-end within a containerised stack.

Experimentally, the K-domination study quantified how the available number of trajectories affects optimisation stability and generalisation for both stages.
The results showed that increasing trajectory coverage reduces both the typical error level and the variability across dataset seeds, and that the residual learning problem in Stage~2 is tightly coupled to the quality of the Stage~1 baseline.
The best-model approach further demonstrated how targeted hyperparameter and feature-mode ablations translate into measurable accuracy--robustness trade-offs.
Finally, the benchmark comparison to the dataset baseline highlighted that the relative performance depends on the operating condition: the proposed physics-informed pipeline is most competitive in regimes where the linear identification baseline exhibits larger errors (e.g., under load and on the larger robot), while the baseline remains difficult to beat in the unloaded UR3 condition.

The results in this thesis motivate several directions for future work.
First, extending the current-domain evaluation to a torque-domain pipeline would allow direct integration with established payload identification methods. This requires converting motor currents to motor torques and, where needed, mapping joint torques to task-space quantities via the Jacobian.
Second, the residual model is currently purely data-driven. Adding explicit, interpretable non-conservative terms (e.g., friction) to the structured dynamics model and using the sequence model only for the remaining error could improve transfer across payloads, trajectories, and robots.
Third, the benchmark results are condition-dependent. Training across multiple datasets and robots, together with domain adaptation and uncertainty-aware model selection, could improve robustness when moving between load conditions and platforms.
Finally, for online use, future work should study incremental adaptation and real-time computational budgets, and evaluate how better inverse-dynamics prediction improves payload parameter identification during manipulation tasks.
