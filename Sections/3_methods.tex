\section{Methods}

Quick overview of what is found in methods (DeLaN + LSTM)-> May start with general DeLaN / LSTM explanation -> moves on with my pipeline mathematics -> then implementation of my pipeline, especially Dataset
and input-/output vector dimentions here, also dataset recording offline training online execution.

Do i have to explain how DeLaN and LSTM work? ->~\cite{Q4_4_10729277,Q4_6_HU2026103093} show DeLaN (also with figures) 
and~\cite{Q3_1_tao_bll,Q3_3_lstm_force_estimation} show LSTM (also with figures). I could also do on figure each, very short explain these two and cite with the given above and DeLaN / LSTM base paper. 

\subsection{Stage~1: Structured inverse dynamics for robot + fixed gripper}

In a first step, we learn a nominal inverse-dynamics model for the
robot--gripper system without payload and without contact.
From the synchronised dataset we obtain
\[
\bigl\{
  \mathbf{q}_k,\dot{\mathbf{q}}_k,\ddot{\mathbf{q}}_k,
  \boldsymbol{I}_k,\vec{F}_{\mathrm{meas},k}
\bigr\}_{k=1}^{N},
\]
where $\mathbf{q}_k,\dot{\mathbf{q}}_k,\ddot{\mathbf{q}}_k \in \mathbb{R}^n$ are
joint position, velocity and acceleration, $\boldsymbol{I}_k \in \mathbb{R}^n$
are motor currents, and $\vec{F}_{\mathrm{meas},k} \in \mathbb{R}^6$ is the
measured flange wrench in the sensor frame $S$.
For brushless DC motors with torque constant $k_t$ \cite{QM_1_ur3_ur5_torque_constant}, the measured motor torques are
\begin{equation}
  \boldsymbol{\tau}_{\mathrm{motor},k} = k_t\,\boldsymbol{I}_k.
\end{equation}

The nominal robot--gripper dynamics are parameterised by a Deep Lagrangian
Network (DeLaN) \cite{Q4_6_HU2026103093} with parameters $\boldsymbol{\theta}$ for the conservative
dynamics and $\boldsymbol{\psi}$ for friction and other non-conservative terms.
The network implements a Lagrangian
\begin{equation}
  \mathcal{L}_{\boldsymbol{\theta}}\bigl(\mathbf{q},\dot{\mathbf{q}}\bigr)
  =
  \tfrac{1}{2}\,\dot{\mathbf{q}}^{\top}
  \mathbf{M}_{\boldsymbol{\theta}}(\mathbf{q})\,\dot{\mathbf{q}}
  -
  V_{\boldsymbol{\theta}}(\mathbf{q}),
\end{equation}
with positive-definite inertia matrix $\mathbf{M}_{\boldsymbol{\theta}}(\mathbf{q})$
(e.g.\ represented via a Cholesky-factor network) and potential
$V_{\boldsymbol{\theta}}(\mathbf{q})$ represented by a neural network.\cite{Q4_1_extended_delan_motor,Q4_4_10729277,Q4_6_HU2026103093}
Using the Euler--Lagrange equations yields the conservative joint torques
\begin{equation}
  \boldsymbol{\tau}_{\mathrm{cons}}
  (\mathbf{q},\dot{\mathbf{q}},\ddot{\mathbf{q}};\boldsymbol{\theta})
  =
  \mathbf{M}_{\boldsymbol{\theta}}(\mathbf{q})\,\ddot{\mathbf{q}}
  +
  \mathbf{C}_{\boldsymbol{\theta}}(\mathbf{q},\dot{\mathbf{q}})\,\dot{\mathbf{q}}
  +
  \mathbf{G}_{\boldsymbol{\theta}}(\mathbf{q}),
\end{equation}
where $\mathbf{C}_{\boldsymbol{\theta}}$ and $\mathbf{G}_{\boldsymbol{\theta}}$ are
implicitly defined by $\mathcal{L}_{\boldsymbol{\theta}}$ \cite{Q4_4_10729277,Q4_6_HU2026103093}.

Joint friction and other non-conservative effects are captured by a
Coulomb--viscous friction model, following the improved DeLaN design
of \cite{Q4_6_HU2026103093,QM_2_Coulomb_viscous_friction_ur5_10610737}.
For each joint $i$ we adopt
\begin{equation}
  \tau_{\mathrm{fric},i}(\dot q_i;\boldsymbol{\psi})
  =
  f_{c,i}\,\operatorname{sgn}(\dot q_i)
  +
  f_{v,i}\,\dot q_i,
\end{equation}
with learned Coulomb and viscous coefficients $f_{c,i}$ and $f_{v,i}$,
collected in $\boldsymbol{\psi}$.
In vector form this can be written compactly as
\begin{equation}
  \boldsymbol{\tau}_{\mathrm{fric}}(\dot{\mathbf{q}};\boldsymbol{\psi})
  =
  f_{\mathrm{fric}}\bigl([\dot{\mathbf{q}},\operatorname{sgn}(\dot{\mathbf{q}})];\boldsymbol{\psi}\bigr),
\end{equation}
where $f_{\mathrm{fric}}$ denotes the joint-wise affine map implementing
the Coulomb--viscous law.

The DeLaN torque prediction is the sum of conservative and frictional parts,
\begin{equation}
  \hat{\boldsymbol{\tau}}_{\mathrm{DeLaN}}
  (\mathbf{q},\dot{\mathbf{q}},\ddot{\mathbf{q}};\boldsymbol{\theta},\boldsymbol{\psi})
  =
  \boldsymbol{\tau}_{\mathrm{cons}}
    (\mathbf{q},\dot{\mathbf{q}},\ddot{\mathbf{q}};\boldsymbol{\theta})
  +
  \boldsymbol{\tau}_{\mathrm{fric}}(\dot{\mathbf{q}};\boldsymbol{\psi}).
\end{equation}
For compactness we write this as a parametric model
\begin{equation}
  \hat{\boldsymbol{\tau}}_{\mathrm{DeLaN}}
  (\mathbf{q},\dot{\mathbf{q}},\ddot{\mathbf{q}};\boldsymbol{\theta},\boldsymbol{\psi})
  =
  f_{\mathrm{DeLaN}}\bigl(\mathbf{q},\dot{\mathbf{q}},\ddot{\mathbf{q}};\boldsymbol{\theta},\boldsymbol{\psi}\bigr),
\end{equation}
so that for sample $k$
\begin{equation}
  \hat{\boldsymbol{\tau}}_{\mathrm{DeLaN},k}
  =
  f_{\mathrm{DeLaN}}\bigl(
    \mathbf{q}_k,\dot{\mathbf{q}}_k,\ddot{\mathbf{q}}_k;
    \boldsymbol{\theta},\boldsymbol{\psi}
  \bigr).
\end{equation}

The parameters $(\boldsymbol{\theta},\boldsymbol{\psi})$ are trained offline by
minimising a joint-space regression loss \cite{Q4_4_10729277,Q4_6_HU2026103093}
\begin{equation}
  \mathcal{L}_{\mathrm{DeLaN}}(\boldsymbol{\theta},\boldsymbol{\psi})
  =
  \frac{1}{N}\sum_{k=1}^{N}
  \left\|
    \hat{\boldsymbol{\tau}}_{\mathrm{DeLaN}}
      (\mathbf{q}_k,\dot{\mathbf{q}}_k,\ddot{\mathbf{q}}_k;
       \boldsymbol{\theta},\boldsymbol{\psi})
    -
    \boldsymbol{\tau}_{\mathrm{motor},k}
  \right\|_2^2.
  \label{eq:delan_joint_loss}
\end{equation}
Note that this training objective uses only joint states and motor torques; the
force/torque sensor is not required for fitting the DeLaN model.

After training, the DeLaN parameters are frozen and the model serves as a
data-driven nominal inverse-dynamics model of the robot--gripper system.

\subsection{Stage~2: Sequence model for residual joint torques}

In the second stage, we model history-dependent effects that are not captured
by the structured DeLaN model, such as backlash and fine-grained nonlinear
friction. Using the same robot--gripper dataset (still without payload and
without contact), we first compute the joint-space residual torques
\begin{equation}
  \boldsymbol{r}_{\tau,k}
  =
  \boldsymbol{\tau}_{\mathrm{motor},k}
  -
  \hat{\boldsymbol{\tau}}_{\mathrm{DeLaN},k}.
  \label{eq:tau_residual}
\end{equation}

Let $H$ denote the sequence length (number of time steps in the history
window). For each time index $k \geq H$ we construct an input sequence
\begin{equation}
  \mathbf{x}_k
  =
  \Bigl[
    \mathbf{q}_{k-H+1:k},\,
    \dot{\mathbf{q}}_{k-H+1:k},\,
    \ddot{\mathbf{q}}_{k-H+1:k},\,
    \hat{\boldsymbol{\tau}}_{\mathrm{DeLaN},k-H+1:k}
  \Bigr],
\end{equation}
where $\mathbf{q}_{a:b}$ denotes the stacked joint vectors
$(\mathbf{q}_a,\dots,\mathbf{q}_b)$, and analogously for
$\dot{\mathbf{q}}$, $\ddot{\mathbf{q}}$ and
$\hat{\boldsymbol{\tau}}_{\mathrm{DeLaN}}$.
This construction stacks the last $H$ joint states together with the
corresponding DeLaN torque predictions into a single sequence feature vector $\mathbf{x}_k$.

An LSTM with parameters $\boldsymbol{\varphi}$ maps this sequence to a
residual-torque prediction
\begin{equation}
  \hat{\boldsymbol{r}}_{\tau,k}
  =
  f_{\mathrm{LSTM}}(\mathbf{x}_k;\boldsymbol{\varphi})
  \in \mathbb{R}^n.
\end{equation}
The LSTM is trained to minimise the mean-squared error between predicted and
true residual torques,
\begin{equation}
  \mathcal{L}_{\mathrm{LSTM}}(\boldsymbol{\varphi})
  =
  \frac{1}{N_H}
  \sum_{k=H}^{N}
  \left\|
    \hat{\boldsymbol{r}}_{\tau,k}
    -
    \boldsymbol{r}_{\tau,k}
  \right\|_2^2,
  \label{eq:lstm_loss}
\end{equation}
where $N_H = N-H+1$ is the number of valid sequences.

The combined joint-space model for the robot--gripper system is then
\begin{equation}
  \hat{\boldsymbol{\tau}}_{\mathrm{RG},k}
  =
  \hat{\boldsymbol{\tau}}_{\mathrm{DeLaN},k}
  +
  \hat{\boldsymbol{r}}_{\tau,k}.
  \label{eq:combined_torque_rg}
\end{equation}
For evaluation in the sensor frame, the corresponding combined flange wrench is
obtained via the Jacobian mapping
\begin{equation}
  \hat{\vec{F}}_{\mathrm{RG},k}
  =
  {}^{S}\!J(\mathbf{q}_k)^{-\top}\,
  \hat{\boldsymbol{\tau}}_{\mathrm{RG},k}.
  \label{eq:combined_wrench_rg}
\end{equation}
Since both stages are trained exclusively on data without payload and without
environment contact, $\hat{\boldsymbol{\tau}}_{\mathrm{RG},k}$ and
$\hat{\vec{F}}_{\mathrm{RG},k}$ represent a high-fidelity, history-aware model
of the nominal robot--gripper dynamics. In later stages, deviations between
this model and the measured joint torques or flange wrenches can be attributed
to the effective rigid-body contribution of additional payloads and contacts.

\begin{figure}[H]
\centering
\includegraphics[width=1\columnwidth]{Images/3_methods/method_figure.drawio.png}
\caption{Two-stage learning pipeline for the robot--gripper nominal dynamics.
Stage~1 learns a structured inverse-dynamics model (DeLaN) in joint space from
encoder and motor-current data and maps its torque predictions to the sensor
frame. Stage~2 uses a sequence model (LSTM) on joint histories and DeLaN
outputs to learn the residual flange wrench, yielding a history-aware
model of the nominal robot--gripper wrench.}
\label{fig:methods_pipeline}
\end{figure}

%%%%%%%%%%%%%%%%%%%%%%%%%%%%%%%%%%%%%%%%%%%%%

