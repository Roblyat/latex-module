\chapter{Background}
    \section{Background}
        \label{sec:Background}

        Linear Acceleration (\(\mathbf{a}\)), Angular Velocity (\(\boldsymbol{\omega}\)), and Angular Acceleration (\(\boldsymbol{\alpha}\)) can be considered independent features with respect to \(\phi_{\text{effective}}\) as long as the robot can perform the motion without exceeding its physical limits. These kinematic parameters are set by the robot's trajectory planning and remain the same whether a payload is carried or not, assuming the payload does not exceed the robot's capabilities. The robot can execute the same trajectory with the same velocity and acceleration parameters regardless of whether it is carrying a payload \cite{urrea2018parameter}.
        
        The motor effort is not independent with respect to \(\phi_{\text{effective}}\) because it is affected by the added inertia and gravitational load when a payload is attached.
        
        The General Joint Torque Equation~\eqref{eq:robot_dynamics} is central to Inverse Dynamics of robots because it determines the joint torques necessary for a given trajectory. In this case, let the robot's rigid body be \(\phi_{\text{robot}}\). Then \(\phi_{\text{robot}} + \phi_{\text{gripper}}\) are reflected as \(\tau_{\text{robot}}\), because both remain the same for every motion \cite{Popov2019}.
        
        \begin{equation}
        \boldsymbol{\tau}_{\text{robot}} = \mathbf{M}(\boldsymbol{q}) \ddot{\boldsymbol{q}} + \mathbf{C}(\boldsymbol{q}, \dot{\boldsymbol{q}}) \dot{\boldsymbol{q}} + \mathbf{G}(\boldsymbol{q})
        \label{eq:robot_dynamics}
        \end{equation}

        External forces like human interaction or handling payloads acting on the robot and influence the joint torques necessary for a given trajectory. So \(\tau_{\text{ext}}\) is the vector of external torques at the joints due to external forces acting on the robot. The transpose of the robots Jacobian matrix \(\mathbf{J}^T\) is used to project these external forces back to the joint space, indicating the torques required at each joint to counteract or respond to those forces. \(\vec{F}_{\text{ext}}(\phi)\) is the vector of external forces applied to the robot's end-effector by a payload. \cite{Popov2019, lu2023external}. 

        \begin{equation}
        \boldsymbol{\tau}_{\text{ext}} = \mathbf{J}^T \vec{F}_{\text{ext}}(\phi)
        \label{eq:tau_ext}
        \end{equation}

        This leads to the motors required torques to move the robots rigid body and a payload for a given trajectory \cite{liu2021sensorless}.
        
        \begin{equation}
        \boldsymbol{\tau}_{\text{motor}} = \boldsymbol{\tau}_{\text{robot}} + \boldsymbol{\tau}_{\text{ext}}(\phi)
        \label{eq:tau_motor}
        \end{equation}

        Since the UR5 robot in this case uses brush less DC motors, the torque-current relationship for electric motors  expresses the linear relationship between the torque \(\tau_{\text{motor}}\) produced by a motor and the current \(\mathbf{I}\) supplied to it \cite{deng2021dynamic,ctms_motor_speed}.
        
        \begin{equation}
        \boldsymbol{\tau}_{\text{motor}} = k_t \cdot \boldsymbol{I} 
        \label{eq:tau_I}
        \end{equation}

        Now converting Equation~\eqref{eq:tau_I} to the motor efforts \(\mathbf{I}\) and inserting Equation~\eqref{eq:tau_motor}
        we see that the joint motor efforts are proportional related to the payload we add at the robots MF, while \(\tau_{\text{robot}}\) remains the same.
        
        \begin{equation}
        \boldsymbol{I} = \frac{\boldsymbol{\tau}_{\text{robot}} + \boldsymbol{\tau}_{\text{ext}}}{k_t}
        \label{eq:I_payload}
        \end{equation}

        The resulting motor torques that are applied to move the robot and the rigid body of the payload result in the f/t measurements on the MF. Where the transmission function from the motor torques to the resulting ft measurements at the MF is non-linear due to the complex interactions within the robot's system. This non-linearity arises from the dynamics of joint coupling, friction, and gear reductions that do not scale linearly with torque. Additionally, the influence of the center of mass and the changing inertia properties as the robot moves contribute to this non-linearity~\eqref{eq:tau_motor} and the external torque projection~\eqref{eq:tau_ext} highlight the non-linear dependencies on the joint angles \(\mathbf{q}\), velocities \(\dot{\mathbf{q}}\), and inertial properties \(\phi\). This relationship can be described with:
        
        \begin{equation}
        \vec{F}_{\text{measured}} = f(\tau_{\text{motor}}, \mathbf{q}, \dot{\mathbf{q}}, \phi)
        \label{eq:f_measured}
        \end{equation}
        
        where \( f(\cdot) \) is a non-linear function that depends on the joint configuration, velocities, and inertial parameters \(\phi\). These factors make the mapping from motor sensor data (effort, position, velocity) to resulting ft measurements at the MF complex and non-linear \cite{liu2019end, blumberg2023estimation}.

        The Motor effort (\(\tau_{\text{motor}}\)~\eqref{eq:tau_motor}) is central to understanding the relationship between motor torques, rigid body motion, and the resulting force/torque (\(\vec{f}\) and \(\vec{\tau}\)) as described by the Newton-Euler equations. \(\tau_{\text{motor}}\) represents the torque produced by the motors, which is applied to move the robot's joints and the connected rigid body. The Newton-Euler equations describe how forces and torques acting on a rigid body are related to its mass properties and motion~\eqref{eq:newtonEuler}. The torque exerted by the motor (\(\tau_{\text{motor}}\)) drives the joints and is responsible for creating motion in the connected rigid body. This torque results in joint accelerations (\(\vec{\alpha}\)), which translate to angular accelerations of the rigid body. The motion described by linear acceleration (\(\vec{a}\)) and angular acceleration (\(\vec{\alpha}\)) depends on the distribution of mass (\(m\)) and the inertia tensor (\(\mathbf{J}_s\)) of the body. When the motors apply more torque (\(\tau_{\text{motor}}\)), they increase the force and torque acting on the rigid body, resulting in higher accelerations (\(\vec{a}\), \(\vec{\alpha}\)) and potentially higher angular velocities (\(\vec{\omega}\)) over time. This means that applying more motor effort translates into faster movement (higher accelerations) and larger force/torque outputs as experienced by the rigid body. The increased motion caused by greater motor effort leads to higher measured force and torque values as captured by an FT sensor mounted on the robot. The Newton-Euler equations show that \(\vec{f}\) and \(\vec{\tau}\) are directly influenced by \(\vec{a}\), \(\vec{\alpha}\), and \(\vec{\omega}\).
        
        When the same rigid body is moved with more effort (more torque from the motors), the result is greater accelerations and velocities. This, in turn, leads to higher force and torque measurements because:
        \begin{equation}
        \vec{f} \propto m\vec{a}, \quad \vec{\tau} \propto \mathbf{J}_s \vec{\alpha} \text{ and } \vec{\tau} \text{ also influenced by } [\vec{\omega}]^{\times} \mathbf{J}_s \vec{\omega}.
        \label{eq:f_proportional}
        \end{equation}
        Hence, higher motor efforts produce stronger outputs in terms of \(\vec{f}\) and \(\vec{\tau}\), as the system's equations reflect this proportional increase.
        
        The joint torques \(\tau_{\text{motor}}\) drive joint motion, leading to increases in \(\vec{a}\), \(\vec{\alpha}\), and \(\vec{\omega}\). The Newton-Euler equations use these motion variables to determine:
        \begin{itemize}
            \item \(\vec{f}\) (force) scales with \(m\vec{a}\).
            \item \(\vec{\tau}\) (torque) scales with \(\mathbf{J}_s \vec{\alpha}\) and includes gyroscopic effects from \([\vec{\omega}]^{\times} \mathbf{J}_s \vec{\omega}\).
        \end{itemize}
        
        Increased effort leads to higher \(\vec{a}\), \(\vec{\alpha}\), and \(\vec{\omega}\), resulting in higher \(\vec{f}\) and \(\vec{\tau}\)~\eqref{eq:f_measured}.