\chapter{Experimental Results}

\textbf{Chapter overview.}
This chapter reports experimental results for the proposed two-stage pipeline, covering (i) the effect of the number of available trajectories ($K$) on DeLaN and residual LSTM training dynamics and accuracy, (ii) a quantitative comparison to the dataset baseline, and (iii) a best-model evaluation on ``with load'' trajectories.

\section{Learning Curve Results}

\subsection{K-Domination Results}

Stage~1 (DeLaN): learning dynamics
Figures~\ref{fig:kdom_delan_train_loss_by_k} and~\ref{fig:kdom_delan_val_mse_by_k}
summarise the Stage~1 optimisation as a function of the number of trajectories
$K$.
Small trajectory sets lead to substantially higher training loss and validation
error, and also to markedly larger variability across dataset seeds.
In contrast, for $K \geq 32$ the curves are closely clustered over the full
training horizon and exhibit similar convergence behaviour.

\begin{figure}[H]
    \centering
    \includegraphics[width=\linewidth]{Images/05_results/k_story/delan/A1_train_loss_by_K.png}
    \caption{DeLaN training loss by $K$ shown as median $\pm$ IQR across dataset seeds (with seed-wise aggregation across DeLaN initialisations).}
    \label{fig:kdom_delan_train_loss_by_k}
\end{figure}

Figure~\ref{fig:kdom_delan_train_loss_by_k} shows that the training loss decreases
rapidly during the first epochs for all $K$ and then continues to decrease more
gradually.
Across the full training horizon, smaller $K$ values remain at higher loss levels
and exhibit wider IQR bands than larger $K$ values.

\begin{figure}[H]
    \centering
    \includegraphics[width=\linewidth]{Images/05_results/k_story/delan/A2_val_mse_by_K.png}
    \caption{DeLaN validation MSE (motor current) by $K$ shown as median $\pm$ IQR across dataset seeds (with seed-wise aggregation across DeLaN initialisations).}
    \label{fig:kdom_delan_val_mse_by_k}
\end{figure}

Figure~\ref{fig:kdom_delan_val_mse_by_k} reports the corresponding validation MSE.
Validation MSE decreases for all $K$ and separates clearly by trajectory count:
smaller $K$ values attain higher validation errors and wider interquartile ranges,
whereas the largest settings concentrate at lower validation MSE.

\textbf{Stage~1 (DeLaN): motor-current error as a function of motion progress.}
Figure~\ref{fig:kdom_delan_rmse_progress_by_k} reports the motor-current RMSE along the
normalised trajectory progress.
The smallest setting ($K=8$) exhibits pronounced error spikes and large IQR,
with peaks exceeding $3\,\mathrm{A}$ in the early part of the motion.
Increasing $K$ substantially reduces both the typical error level and its
variability; for larger $K$, the median curves remain close to each other and the
IQR bands narrow over most of the progress range.

\begin{figure}[H]
    \centering
    \includegraphics[width=0.95\linewidth]{Images/05_results/k_story/delan/A3_torque_rmse_progress_by_K.png}
    \caption{DeLaN motor-current RMSE over normalised progress ($0 \rightarrow 1$) by $K$ shown as median $\pm$ IQR (shaded bands).}
    \label{fig:kdom_delan_rmse_progress_by_k}
\end{figure}

\textbf{Stage~1 (DeLaN): per-joint error.}
Figure~\ref{fig:kdom_delan_rmse_per_joint} reports per-joint motor-current RMSE for all
evaluated values of $K$.
For $K=8$, the error is dominated by joints 3 and 5, where the median RMSE exceeds
$3\,\mathrm{A}$.
For intermediate $K$ (e.g., $K=16$), joint 4 exhibits a pronounced increase in
RMSE relative to the other joints.
For larger $K$, the per-joint medians concentrate within a narrower range and the
IQR bars decrease across joints.

\begin{figure}[H]
    \centering
    \includegraphics[width=0.9\linewidth]{Images/05_results/k_story/delan/A4_torque_rmse_per_joint_grouped.png}
    \caption{DeLaN motor-current RMSE per joint (median $\pm$ IQR) for all evaluated values of $K$ (bars) with IQR error bars.}
    \label{fig:kdom_delan_rmse_per_joint}
\end{figure}

\textbf{Stage~2 (LSTM): learning dynamics.}
Figures~\ref{fig:kdom_lstm_train_loss_by_k} and~\ref{fig:kdom_lstm_val_loss_by_k}
show the Stage~2 training and validation loss of the residual LSTM.
While the training loss decreases rapidly for all $K$, the validation loss shows
a clear dependence on the number of trajectories: larger $K$ yields lower
validation loss and narrower IQR bands across the training horizon.

\begin{figure}[H]
    \centering
    \includegraphics[width=0.9\linewidth]{Images/05_results/k_story/lstm/B1_train_loss_by_K.png}
    \caption{LSTM training loss by $K$ shown as median $\pm$ IQR across dataset seeds ($H=100$, feature mode \texttt{full}).}
    \label{fig:kdom_lstm_train_loss_by_k}
\end{figure}

\begin{figure}[H]
    \centering
    \includegraphics[width=0.9\linewidth]{Images/05_results/k_story/lstm/B2_val_loss_by_K.png}
    \caption{LSTM validation loss by $K$ shown as median $\pm$ IQR across dataset seeds ($H=100$, feature mode \texttt{full}).}
    \label{fig:kdom_lstm_val_loss_by_k}
\end{figure}

\textbf{Stage~2 (LSTM): residual error.}
Figure~\ref{fig:kdom_lstm_res_rmse_progress_by_k} shows the residual motor-current RMSE over
normalised trajectory progress, and Figure~\ref{fig:kdom_lstm_res_rmse_per_joint}
summarises per-joint residual errors.
The smallest setting ($K=8$) exhibits large residual spikes and wide IQR bands,
whereas larger $K$ values concentrate at substantially smaller residual RMSE
levels across progress and across joints.

\begin{figure}[H]
    \centering
    \includegraphics[width=0.9\linewidth]{Images/05_results/k_story/lstm/B4_residual_rmse_per_joint_grouped.png}
    \caption{LSTM residual motor-current RMSE per joint (median $\pm$ IQR) for selected values of $K$ ($H=100$, feature mode \texttt{full}).}
    \label{fig:kdom_lstm_res_rmse_per_joint}
\end{figure}

\begin{figure}[H]
    \centering
    \includegraphics[width=0.95\linewidth]{Images/05_results/k_story/lstm/B3_residual_rmse_progress_by_K.png}
    \caption{LSTM residual motor-current RMSE over normalised progress ($0 \rightarrow 1$) by $K$ shown as median $\pm$ IQR (shaded bands) ($H=100$, feature mode \texttt{full}).}
    \label{fig:kdom_lstm_res_rmse_progress_by_k}
\end{figure}

\textbf{End-to-end motor-current accuracy (per joint).}
Figure~\ref{fig:kdom_combined_rmse_per_joint_example} reports per-joint motor-current
RMSE on the test split for a representative evaluation (valid indices $k\geq H-1$).
Across all joints, the combined DeLaN+LSTM predictor attains lower RMSE than the
DeLaN baseline, with the largest absolute differences observed on joints 0--2.

\begin{figure}[H]
    \centering
    \includegraphics[width=\linewidth]{Images/05_results/k_story/evaluation/torque_rmse_per_joint_grouped_test_H100.png}
    \caption{Representative per-joint motor-current RMSE on the test split comparing DeLaN and the combined DeLaN+LSTM predictor (valid indices $k\geq H-1$, $H=100$).}
    \label{fig:kdom_combined_rmse_per_joint_example}
\end{figure}

\subsection{Best Model Approach Results}

\textbf{DeLaN Best Model}
Figures~\ref{fig:delan_train_val_loss_across_seed_by_hyper}--\ref{fig:delan_rmse_per_joint}
summarise the DeLaN best-model sweep at fixed $K=84$.
The results are aggregated across dataset seeds with seed-wise aggregation across
DeLaN initialisations.

\begin{figure}[H]
    \centering
    \includegraphics[width=\linewidth]{Images/05_results/best_model_story/delan/delan_best_train_val_curves_by_hp.png}
    \caption{DeLaN training loss (left) and validation MSE in motor-current units (right) shown as median $\pm$ IQR across seeds for each hyperparameter preset.}
    \label{fig:delan_train_val_loss_across_seed_by_hyper}
\end{figure}

Figure~\ref{fig:delan_train_val_loss_across_seed_by_hyper} shows that all presets
exhibit a rapid initial decrease in training loss and validation MSE, followed by
a slower decay over the remaining epochs.
Across presets, the curves separate consistently both in training loss and in the
final validation MSE plateau.

\begin{figure}[H]
    \centering
    \includegraphics[width=\linewidth]{Images/05_results/best_model_story/delan/scatter_accuracy_vs_stability.png}
    \caption{Hyperparameter comparison: median validation motor-current RMSE versus seed-stability measured by the median within-split IQR of validation RMSE (each point denotes one DeLaN hyperparameter preset; statistics aggregated across dataset seeds with seed-wise aggregation across DeLaN initialisations).}
    \label{fig:delan_scatter_val_vs_valIQR}
\end{figure}

Figure~\ref{fig:delan_scatter_val_vs_valIQR} reports one point per preset in the
plane of (median validation RMSE, median within-split IQR of validation RMSE).
The points span a validation RMSE range of approximately $0.309$--$0.330\,\mathrm{A}$
and a stability range of approximately $0.009$--$0.023\,\mathrm{A}$.

\begin{figure}[H]
    \centering
    \includegraphics[width=\linewidth]{Images/05_results/best_model_story/delan/scatter_val_vs_test.png}
    \caption{Hyperparameter comparison: median validation motor-current RMSE versus median test motor-current RMSE (each point denotes one DeLaN hyperparameter preset; statistics aggregated across dataset seeds with seed-wise aggregation across DeLaN initialisations).}
    \label{fig:delan_scatter_val_vs_test}
\end{figure}

Figure~\ref{fig:delan_scatter_val_vs_test} shows the corresponding (median validation RMSE,
median test RMSE) pairs for each preset, with the observed points spanning roughly
$0.309$--$0.330\,\mathrm{A}$ on the validation axis and $0.280$--$0.340\,\mathrm{A}$
on the test axis.

\begin{figure}[H]
    \centering
    \includegraphics[width=\linewidth]{Images/05_results/best_model_story/delan/A4_torque_rmse_per_joint_grouped.png}
    \caption{DeLaN motor-current RMSE per joint (median $\pm$ IQR) grouped by hyperparameter preset.}
    \label{fig:delan_rmse_per_joint}
\end{figure}

Figure~\ref{fig:delan_rmse_per_joint} reports per-joint motor-current RMSE by preset.
Across presets, joint 1 exhibits the largest median RMSE, followed by joints 2 and 4,
whereas joint 5 attains the lowest median RMSE.
The IQR bars are largest on joints 1 and 4.

%%%%%%%%%%%%%%%%%%%%%%%%%%%%%%%%%%%%%%%%%%%%%%%%%%%%%%%%%%%%%%%%%%%%%%%%%%%%%%%%%%%%%%%%%%%%%%%%%%%%%%%%%%%%%%%%%%%%%%%%%%%%%%%%%%%%%%%%%%%%%%%%%%%%%%%%%%%%%%%%%%%%%%%%%%%%%%%%%%%%

\textbf{LSTM Best Model}
Figures~\ref{fig:lstm_val_loss_vs_rmse_total}--\ref{fig:lstm_residual_gt_vs_predicted}
summarise the LSTM best-model sweep at fixed $K=84$ based on the residual datasets exported
from the selected DeLaN baseline.

\begin{figure}[H]
    \centering
    \includegraphics[width=\linewidth]{Images/05_results/best_model_story/lstm/lstm_best_val_loss_vs_rmse_total.png}
    \caption{Best validation loss versus total motor-current RMSE on the test split for the LSTM residual model (marker colour denotes feature mode; marker outline denotes history length $H$).}
    \label{fig:lstm_val_loss_vs_rmse_total}
\end{figure}

Figure~\ref{fig:lstm_val_loss_vs_rmse_total} shows that most configurations cluster at low
test RMSE (approximately $0.165$--$0.195\,\mathrm{A}$) and low best validation loss
(approximately $0.07$--$0.18$), while a smaller set of configurations attains markedly higher
test RMSE values around $0.31$--$0.34\,\mathrm{A}$.

\begin{figure}[H]
    \centering
    \includegraphics[width=\linewidth]{Images/05_results/best_model_story/lstm/lstm_overfit_ratio_by_feature_mode.png}
    \caption{Overfit indicator (final validation loss / final training loss) by feature mode shown as boxplots across runs (green triangles indicate means; panels use different $y$-axis ranges for readability).}
    \label{fig:lstm_overfit_ratio}
\end{figure}

Figure~\ref{fig:lstm_overfit_ratio} reports the distribution of the final validation-to-training
loss ratio across feature modes.
Across the lower-range panel, the medians lie around $5$--$10$ depending on the mode, while
the high-range panel includes ratios extending beyond $30$.

\begin{figure}[H]
    \centering
    \includegraphics[width=\linewidth]{Images/05_results/best_model_story/lstm/lstm_residual_rmse_by_feature_mode.png}
    \caption{LSTM residual motor-current RMSE on the test split by feature mode shown as boxplots across runs (green triangles indicate means; panels use different $y$-axis ranges for readability).}
    \label{fig:lstm_residual_rmse_by_feature_mode}
\end{figure}

Figure~\ref{fig:lstm_residual_rmse_by_feature_mode} reports residual RMSE on the test split by
feature mode.
In the lower-range panel, the residual RMSE distributions are concentrated around
$0.17$--$0.19\,\mathrm{A}$, whereas the high-range panel contains configurations with residual
RMSE around $0.31$--$0.37\,\mathrm{A}$.

\begin{figure}[H]
    \centering
    \includegraphics[width=\linewidth]{Images/05_results/best_model_story/lstm/lstm_residual_rmse_by_H.png}
    \caption{LSTM residual motor-current RMSE on the test split shown as boxplots across history lengths $H$ (green triangles indicate means).}
    \label{fig:lstm_residual_rmse_by_window_length}
\end{figure}

Figure~\ref{fig:lstm_residual_rmse_by_window_length} shows residual RMSE grouped by history
length $H$.
Across $H\in\{50,100,150\}$, the medians lie in a similar range and the boxplots exhibit
comparable spread, with outliers present for all three history lengths.

\begin{figure}[H]
    \centering
    \includegraphics[width=\linewidth]{Images/05_results/best_model_story/lstm/residual_gt_vs_pred_test_H150.png}
    \caption{Representative residual motor-current traces: ground-truth residual versus LSTM residual prediction on the test split (valid indices $k\geq H-1$, $H=150$).}
    \label{fig:lstm_residual_gt_vs_predicted}
\end{figure}

Figure~\ref{fig:lstm_residual_gt_vs_predicted} provides a representative residual overlay for
all joints, showing the ground-truth residual and the corresponding LSTM prediction over time.

%%%%%%%%%%%%%%%%%%%%%%%%%%%%%%%%%%%%%%%%%%%%%%%%%%%%%%%%%%%%%%%%%%%%%%%%%%%%%%%%%%%%%%%%%%%%%%%%%%%%%%%%%%%%%%%%%%%%%%%%%%%%%%%%%%%%%%%%%%%%%%%%%%%%%%%%%%%%%%%%%%%%%%%%%%%%%%%%%%%%

\textbf{Best Pipeline}

Figure~\ref{fig:kdom_residual_overlay_example} reports the per-joint motor-current RMSE on the test
split for the selected DeLaN baseline and for the combined DeLaN+LSTM predictor (valid indices
$k\geq H-1$, $H=150$).
Across joints, the DeLaN bars range from approximately $0.12$ to $0.46\,\mathrm{A}$, whereas the
combined predictor ranges from approximately $0.07$ to $0.25\,\mathrm{A}$.
The largest RMSE is observed on joint 1 and the smallest on joint 5 for both models.

\begin{figure}[H]
    \centering
    \includegraphics[width=\linewidth]{Images/05_results/best_model_story/evaluation/torque_rmse_per_joint_grouped_test_H150.png}
    \caption{Best pipeline motor-current RMSE per joint on the test split comparing DeLaN and the combined DeLaN+LSTM predictor (valid indices $k\geq H-1$, $H=150$).}
    \label{fig:kdom_residual_overlay_example}
\end{figure}

Figure~\ref{fig:kdom_torque_overlay_example} shows the corresponding motor-current time-series
overlay for the same evaluation, plotting the ground truth together with the DeLaN prediction and
the combined DeLaN+LSTM prediction for each joint.

\begin{figure}[H]
    \centering
    \includegraphics[width=\linewidth]{Images/05_results/best_model_story/evaluation/torque_gt_vs_delan_vs_combined_test_H150.png}
    \caption{Representative motor-current overlay: ground truth, DeLaN prediction, and combined DeLaN+LSTM prediction on the test split (valid indices $k\geq H-1$, $H=150$).}
    \label{fig:kdom_torque_overlay_example}
\end{figure}

\section{Performance Evaluation to Baseline}

\textbf{5.3 Performance Evaluation to Baseline.}
To benchmark the proposed DeLaN+LSTM pipeline against prior work on the same dataset,
we compare against the data-driven dynamic model presented in~\cite{Q_Dataset_Paper}
and its accompanying IEEE DataPort release~\cite{Dataset_Dataport}.
The baseline study likewise applies low-pass filtering to the recorded signals as part of its preprocessing~\cite{Q_Dataset_Paper}.
The baseline identifies a joint-wise, linear-in-parameters model of the motor current $i_i$
from measured joint positions, velocities, and accelerations, and estimates the unknown
parameter vector via least squares:
\begin{equation}
  \mathbf{K}_i
  =
  \left(\boldsymbol{\Theta}_{m,i}^\top \boldsymbol{\Theta}_{m,i}\right)^{-1}
  \boldsymbol{\Theta}_{m,i}^\top \mathbf{i}_{m,i},
\end{equation}
where $\boldsymbol{\Theta}_{m,i}$ is the measurement matrix of the regressor terms for joint $i$
and $\mathbf{i}_{m,i}$ denotes the corresponding measured motor-current samples~\cite{Q_Dataset_Paper}.
Evaluation is reported as per-joint RMSE in current units,
\begin{equation}
  \mathrm{RMSE}(i_i)
  =
  \sqrt{\frac{1}{n}\sum_{k=1}^{n}\left(i_{i,k}^{\mathrm{real}}-i_{i,k}^{\mathrm{pred}}\right)^2}.
\end{equation}

\textbf{Baseline results on IEEE DataPort dataset.}
Tables~\ref{tab:baseline_current_rmse_train} and~\ref{tab:baseline_current_rmse_test}
reproduce the current-prediction RMSE reported by~\cite{Q_Dataset_Paper} for UR3e and UR10e,
each without load and with load, for the provided training and testing datasets.
Across conditions, UR3e baseline RMSE values are on the order of $0.06$--$0.38\,\mathrm{A}$,
while UR10e baseline RMSE values range up to $1.56\,\mathrm{A}$ on the test set.

\begin{table}[H]
    \centering
    \scriptsize
    \setlength{\tabcolsep}{3pt}
    \caption{Baseline motor-current RMSE on the training dataset (Table~I in~\cite{Q_Dataset_Paper}).}
    \label{tab:baseline_current_rmse_train}
    \resizebox{\linewidth}{!}{%
    \begin{tabular}{l r r r r r r r r r r r r}
        \toprule
        RMSE & $i_1$ [A] & \% & $i_2$ [A] & \% & $i_3$ [A] & \% & $i_4$ [A] & \% & $i_5$ [A] & \% & $i_6$ [A] & \% \\
        \midrule
        UR3e without load & 0.103 & 1.79 & 0.118 & 1.72 & 0.114 & 2.77 & 0.071 & 3.26 & 0.077 & 3.79 & 0.063 & 3.11 \\
        UR3e with load    & 0.101 & 1.76 & 0.182 & 2.49 & 0.154 & 3.43 & 0.136 & 5.32 & 0.092 & 4.50 & 0.071 & 3.47 \\
        UR10e without load& 0.477 & 1.88 & 0.615 & 0.85 & 0.322 & 1.28 & 0.096 & 3.27 & 0.091 & 3.57 & 0.096 & 3.34 \\
        UR10e with load   & 0.444 & 1.38 & 0.692 & 0.96 & 0.338 & 1.34 & 0.104 & 3.35 & 0.089 & 3.47 & 0.087 & 3.02 \\
        \bottomrule
    \end{tabular}%
    }
\end{table}

\begin{table}[H]
    \centering
    \scriptsize
    \setlength{\tabcolsep}{3pt}
    \caption{Baseline motor-current RMSE on the testing dataset (Table~II in~\cite{Q_Dataset_Paper}).}
    \label{tab:baseline_current_rmse_test}
    \resizebox{\linewidth}{!}{%
    \begin{tabular}{l r r r r r r r r r r r r}
        \toprule
        RMSE & $i_1$ [A] & \% & $i_2$ [A] & \% & $i_3$ [A] & \% & $i_4$ [A] & \% & $i_5$ [A] & \% & $i_6$ [A] & \% \\
        \midrule
        UR3e without load & 0.089 & 3.14 & 0.130 & 2.87 & 0.122 & 4.55 & 0.066 & 4.39 & 0.106 & 8.29 & 0.083 & 5.89 \\
        UR3e with load    & 0.096 & 3.31 & 0.376 & 5.63 & 0.308 & 5.09 & 0.249 & 8.53 & 0.150 & 8.10 & 0.070 & 4.10 \\
        UR10e without load& 0.490 & 3.24 & 0.853 & 2.54 & 0.351 & 3.14 & 0.110 & 4.40 & 0.098 & 6.48 & 0.078 & 5.12 \\
        UR10e with load   & 0.809 & 4.39 & 1.555 & 6.05 & 0.653 & 5.06 & 0.185 & 9.88 & 0.097 & 6.47 & 0.080 & 5.41 \\
        \bottomrule
    \end{tabular}%
    }
\end{table}

\textbf{Our evaluation protocol and split.}
In addition to the dataset baseline, we report motor-current RMSE for the DeLaN and DeLaN+LSTM
models produced by the thesis-internal model-selection procedure at fixed $K=84$ trajectories and
split sizes (train, val, test) = (59, 8, 17) trajectories.
Since the dataset provides motor current measurements, and motor current is treated as the primary
actuation signal throughout this thesis, we report RMSE in current units [A] per joint.
Load-condition baselines are included as placeholders and will be filled once the best-model pipeline
is re-run on the ``with load'' trajectories.

\begin{table}[H]
    \centering
    \caption{DeLaN motor-current RMSE per joint on the held-out test split (17 trajectories, $K=84$).}
    \label{tab:baseline_ours_delan_test}
    \begin{tabular}{l r r r r r r}
        \toprule
        & $i_1$ [A] & $i_2$ [A] & $i_3$ [A] & $i_4$ [A] & $i_5$ [A] & $i_6$ [A] \\
        \midrule
        UR3e without load & 0.368 & 0.405 & 0.286 & 0.226 & 0.219 & 0.202 \\
        UR3e with load (\textit{TBD}) & -- & -- & -- & -- & -- & -- \\
        \bottomrule
    \end{tabular}
\end{table}

\begin{table}[H]
    \centering
    \caption{DeLaN motor-current RMSE per joint on the validation split (8 trajectories, $K=84$).}
    \label{tab:baseline_ours_delan_val}
    \begin{tabular}{l r r r r r r}
        \toprule
        & $i_1$ [A] & $i_2$ [A] & $i_3$ [A] & $i_4$ [A] & $i_5$ [A] & $i_6$ [A] \\
        \midrule
        UR3e without load & 0.408 & 0.484 & 0.331 & 0.284 & 0.273 & 0.257 \\
        UR3e with load (\textit{TBD}) & -- & -- & -- & -- & -- & -- \\
        \bottomrule
    \end{tabular}
\end{table}

\begin{table}[H]
    \centering
    \caption{Best-model DeLaN+LSTM pipeline motor-current RMSE per joint on the validation split (8 trajectories, valid indices $k\geq H-1$).}
    \label{tab:baseline_ours_fullpipeline_val}
    \begin{tabular}{l r r r r r r}
        \toprule
        & $i_1$ [A] & $i_2$ [A] & $i_3$ [A] & $i_4$ [A] & $i_5$ [A] & $i_6$ [A] \\
        \midrule
        UR3e without load & 0.069 & 0.083 & 0.124 & 0.063 & 0.086 & 0.052 \\
        UR3e with load (\textit{TBD}) & -- & -- & -- & -- & -- & -- \\
        \bottomrule
    \end{tabular}
\end{table}

\textbf{Reporting notes.}
Direct numerical comparison to~\cite{Q_Dataset_Paper} is not one-to-one, since the
baseline is identified on the dataset-provided training split (50k samples) and evaluated on the
dataset-provided test split (5k samples), whereas our results are obtained from the thesis-specific
trajectory-wise split (59/8/17 trajectories) and include the LSTM warm-up constraint $k\geq H-1$.
Accordingly, Tables~\ref{tab:baseline_ours_delan_test}--\ref{tab:baseline_ours_fullpipeline_val}
report results under the thesis-specific split and evaluation constraints.

\section{DeLaN+LSTM Gripper Compensation}

\textbf{Best-model evaluation on ``with load'' trajectories.}
Following the best-model selection procedure, we evaluate the final DeLaN+LSTM pipeline on the
``with load'' trajectories of the IEEE DataPort dataset~\cite{Dataset_Dataport} and report the
corresponding motor-current RMSE metrics.
The corresponding best-model metrics are reported in Table~\ref{tab:gripper_comp_table7_delan} and Table~\ref{tab:gripper_comp_table8_fullpipeline}
(placeholders shown until the best-model run on the loaded dataset is completed).

\begin{table}[H]
    \centering
    \scriptsize
    \setlength{\tabcolsep}{3pt}
    \caption{Best-model DeLaN motor-current RMSE per joint for gripper compensation on dataset~\cite{Dataset_Dataport} (placeholders).}
    \label{tab:gripper_comp_table7_delan}
    \begin{tabular}{l r r r r r r}
        \toprule
        Condition & $i_1$ [A] & $i_2$ [A] & $i_3$ [A] & $i_4$ [A] & $i_5$ [A] & $i_6$ [A] \\
        \midrule
        UR3e without load (\textit{TBD}) & -- & -- & -- & -- & -- & -- \\
        UR3e with load (\textit{TBD})    & -- & -- & -- & -- & -- & -- \\
        \bottomrule
    \end{tabular}
\end{table}

\begin{table}[H]
    \centering
    \scriptsize
    \setlength{\tabcolsep}{3pt}
    \caption{Best-model DeLaN+LSTM pipeline motor-current RMSE per joint for gripper compensation on dataset~\cite{Dataset_Dataport} (placeholders).}
    \label{tab:gripper_comp_table8_fullpipeline}
    \begin{tabular}{l r r r r r r}
        \toprule
        Condition & $i_1$ [A] & $i_2$ [A] & $i_3$ [A] & $i_4$ [A] & $i_5$ [A] & $i_6$ [A] \\
        \midrule
        UR3e without load (\textit{TBD}) & -- & -- & -- & -- & -- & -- \\
        UR3e with load (\textit{TBD})    & -- & -- & -- & -- & -- & -- \\
        \bottomrule
    \end{tabular}
\end{table}
