\chapter{Experimental Results}

This chapter reports experimental results for the proposed two-stage pipeline, covering the effect of the number of available 
trajectories ($K$) on DeLaN and residual LSTM training dynamics and accuracy. A quantitative comparison to the dataset baseline 
and a best-model evaluation on ``with load'' and ``without load'' trajectories.

\section{Learning Curve Results}

This section reports learning-curve experiments in which the number of available trajectories
$K$ is varied to quantify how trajectory coverage affects optimisation dynamics and
motor-current prediction accuracy for both stages of the pipeline.
Across the presented losses and RMSE metrics, smaller $K$ values exhibit higher dispersion
across dataset seeds and more pronounced error variability, while larger $K$ values yield more
consistent convergence behaviour and lower typical RMSE.
The final comparison figures additionally report the per-joint test RMSE for DeLaN versus the
combined DeLaN+LSTM predictor for representative runs.

\subsection{K-Domination Results}

Stage~1 (DeLaN): learning dynamics
Figures~\ref{fig:kdom_delan_train_loss_by_k} and~\ref{fig:kdom_delan_val_mse_by_k}
summarise the Stage~1 optimisation as a function of the number of trajectories
$K$.
Small trajectory sets lead to substantially higher training loss and validation
error, and also to markedly larger variability across dataset seeds.
In contrast, for $K \geq 32$ the curves are closely clustered over the full
training horizon and exhibit similar convergence behaviour.

\begin{figure}[H]
    \centering
    \includegraphics[width=\linewidth]{Images/05_results/k_story/delan/A1_train_loss_by_K.png}
    \caption{DeLaN training loss by $K$ shown as median $\pm$ IQR across dataset seeds (with seed-wise aggregation across DeLaN initialisations).}
    \label{fig:kdom_delan_train_loss_by_k}
\end{figure}

Figure~\ref{fig:kdom_delan_train_loss_by_k} shows that the training loss decreases
rapidly during the first epochs for all $K$ and then continues to decrease more
gradually.
Across the full training horizon, smaller $K$ values remain at higher loss levels
and exhibit wider IQR bands than larger $K$ values.

\begin{figure}[H]
    \centering
    \includegraphics[width=\linewidth]{Images/05_results/k_story/delan/A2_val_mse_by_K.png}
    \caption{DeLaN validation MSE (motor current) by $K$ shown as median $\pm$ IQR across dataset seeds (with seed-wise aggregation across DeLaN initialisations).}
    \label{fig:kdom_delan_val_mse_by_k}
\end{figure}

Figure~\ref{fig:kdom_delan_val_mse_by_k} reports the corresponding validation MSE.
Validation MSE decreases for all $K$ and separates clearly by trajectory count:
smaller $K$ values attain higher validation errors and wider interquartile ranges,
whereas the largest settings concentrate at lower validation MSE.

Figure~\ref{fig:kdom_delan_rmse_progress_by_k} reports the motor-current RMSE along the
normalised trajectory progress.
The smallest setting ($K=8$) exhibits pronounced error spikes and large IQR,
with peaks exceeding $3\,\mathrm{A}$ in the early part of the motion.
Increasing $K$ substantially reduces both the typical error level and its
variability. For larger $K$, the median curves remain close to each other and the
IQR bands narrow over most of the progress range.

\begin{figure}[H]
    \centering
    \includegraphics[width=0.95\linewidth]{Images/05_results/k_story/delan/A3_torque_rmse_progress_by_K.png}
    \caption{DeLaN motor-current RMSE over normalised progress ($0 \rightarrow 1$) by $K$ shown as median $\pm$ IQR (shaded bands).}
    \label{fig:kdom_delan_rmse_progress_by_k}
\end{figure}

Figure~\ref{fig:kdom_delan_rmse_per_joint} reports per-joint motor-current RMSE for all
evaluated values of $K$.
For $K=8$, the error is dominated by joints 3 and 5, where the median RMSE exceeds
$3\,\mathrm{A}$.
For intermediate $K$ (like $K=16$), joint 4 exhibits a pronounced increase in
RMSE relative to the other joints.
For larger $K$, the per-joint medians concentrate within a narrower range and the
IQR bars decrease across joints.

\begin{figure}[H]
    \centering
    \includegraphics[width=0.9\linewidth]{Images/05_results/k_story/delan/A4_torque_rmse_per_joint_grouped.png}
    \caption{DeLaN motor-current RMSE per joint (median $\pm$ IQR) for all evaluated values of $K$ (bars) with IQR error bars.}
    \label{fig:kdom_delan_rmse_per_joint}
\end{figure}

Figures~\ref{fig:kdom_lstm_train_loss_by_k} and~\ref{fig:kdom_lstm_val_loss_by_k}
show the Stage~2 training and validation loss of the residual LSTM.
While the training loss decreases rapidly for all $K$, the validation loss shows
a clear dependence on the number of trajectories: larger $K$ yields lower
validation loss and narrower IQR bands across the training horizon.

\begin{figure}[H]
    \centering
    \includegraphics[width=0.9\linewidth]{Images/05_results/k_story/lstm/B1_train_loss_by_K.png}
    \caption{LSTM training loss by $K$ shown as median $\pm$ IQR across dataset seeds ($H=100$, feature mode \texttt{full}).}
    \label{fig:kdom_lstm_train_loss_by_k}
\end{figure}

\begin{figure}[H]
    \centering
    \includegraphics[width=0.9\linewidth]{Images/05_results/k_story/lstm/B2_val_loss_by_K.png}
    \caption{LSTM validation loss by $K$ shown as median $\pm$ IQR across dataset seeds ($H=100$, feature mode \texttt{full}).}
    \label{fig:kdom_lstm_val_loss_by_k}
\end{figure}

Figure~\ref{fig:kdom_lstm_res_rmse_progress_by_k} shows the residual motor-current RMSE over
normalised trajectory progress, and Figure~\ref{fig:kdom_lstm_res_rmse_per_joint}
summarises per-joint residual errors.
The smallest setting ($K=8$) exhibits large residual spikes and wide IQR bands,
whereas larger $K$ values concentrate at substantially smaller residual RMSE
levels across progress and across joints.

\begin{figure}[H]
    \centering
    \includegraphics[width=0.9\linewidth]{Images/05_results/k_story/lstm/B4_residual_rmse_per_joint_grouped.png}
    \caption{LSTM residual motor-current RMSE per joint (median $\pm$ IQR) for selected values of $K$ ($H=100$, feature mode \texttt{full}).}
    \label{fig:kdom_lstm_res_rmse_per_joint}
\end{figure}

\begin{figure}[H]
    \centering
    \includegraphics[width=0.95\linewidth]{Images/05_results/k_story/lstm/B3_residual_rmse_progress_by_K.png}
    \caption{LSTM residual motor-current RMSE over normalised progress ($0 \rightarrow 1$) by $K$ shown as median $\pm$ IQR (shaded bands) ($H=100$, feature mode \texttt{full}).}
    \label{fig:kdom_lstm_res_rmse_progress_by_k}
\end{figure}

Figure~\ref{fig:kdom_combined_rmse_per_joint_example} reports per-joint motor-current
RMSE on the test split for a representative evaluation (valid indices $k\geq H-1$).
Across all joints, the combined DeLaN+LSTM predictor attains lower RMSE than the
DeLaN baseline, with the largest absolute differences observed on joints 0--2.

\begin{figure}[H]
    \centering
    \includegraphics[width=\linewidth]{Images/05_results/k_story/evaluation/torque_rmse_per_joint_grouped_test_H100.png}
    \caption{Representative per-joint motor-current RMSE on the test split comparing DeLaN and the combined DeLaN+LSTM predictor (valid indices $k\geq H-1$, $H=100$).}
    \label{fig:kdom_combined_rmse_per_joint_example}
\end{figure}

\subsection{Best Model Approach Results}

Figures~\ref{fig:delan_train_val_loss_across_seed_by_hyper}--\ref{fig:delan_rmse_per_joint}
summarise the DeLaN best-model sweep at fixed $K=84$ on the $50{,}000$-sample
\texttt{UR3\_Load0Dataset} training pool.
All metrics in this subsection are computed on this pool, which is split into $70\%$ training,
$10\%$ validation, and $20\%$ test subsets for model selection.
Results are aggregated across dataset seeds with seed-wise aggregation across DeLaN initialisations.

\begin{figure}[H]
    \centering
    \includegraphics[width=\linewidth]{Images/05_results/best_model_story/delan/delan_best_train_val_curves_by_hp.png}
    \caption{DeLaN training loss (left) and validation MSE in motor-current units (right) shown as median $\pm$ IQR across seeds for each hyperparameter preset.}
    \label{fig:delan_train_val_loss_across_seed_by_hyper}
\end{figure}

Figure~\ref{fig:delan_train_val_loss_across_seed_by_hyper} shows that all presets
exhibit a rapid initial decrease in training loss and validation MSE, followed by
a slower decay over the remaining epochs.
Across presets, the curves separate consistently both in training loss and in the
final validation MSE plateau.

\begin{figure}[H]
    \centering
    \includegraphics[width=\linewidth]{Images/05_results/best_model_story/delan/scatter_accuracy_vs_stability.png}
    \caption{Hyperparameter comparison: median validation motor-current RMSE versus seed-stability measured by the median within-split IQR of validation RMSE (each point denotes one DeLaN hyperparameter preset. Statistics aggregated across dataset seeds with seed-wise aggregation across DeLaN initialisations).}
    \label{fig:delan_scatter_val_vs_valIQR}
\end{figure}

Figure~\ref{fig:delan_scatter_val_vs_valIQR} reports one point per preset in the
plane of (median validation RMSE, median within-split IQR of validation RMSE).
The points span a validation RMSE range of approximately $0.309$--$0.330\,\mathrm{A}$
and a stability range of approximately $0.009$--$0.023\,\mathrm{A}$.

\begin{figure}[H]
    \centering
    \includegraphics[width=\linewidth]{Images/05_results/best_model_story/delan/scatter_val_vs_test.png}
    \caption{Hyperparameter comparison: median validation motor-current RMSE versus median test motor-current RMSE (each point denotes one DeLaN hyperparameter preset. Statistics aggregated across dataset seeds with seed-wise aggregation across DeLaN initialisations).}
    \label{fig:delan_scatter_val_vs_test}
\end{figure}

Figure~\ref{fig:delan_scatter_val_vs_test} shows the corresponding (median validation RMSE,
median test RMSE) pairs for each preset, with the observed points spanning roughly
$0.309$--$0.330\,\mathrm{A}$ on the validation axis and $0.280$--$0.340\,\mathrm{A}$
on the test axis.

\begin{figure}[H]
    \centering
    \includegraphics[width=\linewidth]{Images/05_results/best_model_story/delan/A4_torque_rmse_per_joint_grouped.png}
    \caption{DeLaN motor-current RMSE per joint (median $\pm$ IQR) grouped by hyperparameter preset.}
    \label{fig:delan_rmse_per_joint}
\end{figure}

Figure~\ref{fig:delan_rmse_per_joint} reports per-joint motor-current RMSE by preset.
Across presets, joint 1 exhibits the largest median RMSE, followed by joints 2 and 4,
whereas joint 5 attains the lowest median RMSE.
The IQR bars are largest on joints 1 and 4.

%%%%%%%%%%%%%%%%%%%%%%%%%%%%%%%%%%%%%%%%%%%%%%%%%%%%%%%%%%%%%%%%%%%%%%%%%%%%%%%%%%%%%%%%%%%%%%%%%%%%%%%%%%%%%%%%%%%%%%%%%%%%%%%%%%%%%%%%%%%%%%%%%%%%%%%%%%%%%%%%%%%%%%%%%%%%%%%%%%%%

Figures~\ref{fig:lstm_val_loss_vs_rmse_total}--\ref{fig:lstm_residual_gt_vs_predicted}
summarise the LSTM best-model sweep at fixed $K=84$ based on the residual datasets exported
from the selected DeLaN baseline.

\begin{figure}[H]
    \centering
    \includegraphics[width=\linewidth]{Images/05_results/best_model_story/lstm/lstm_best_val_loss_vs_rmse_total.png}
    \caption{Best validation loss versus total motor-current RMSE on the test split for the LSTM residual model (marker colour denotes feature mode, marker outline denotes history length $H$).}
    \label{fig:lstm_val_loss_vs_rmse_total}
\end{figure}

Figure~\ref{fig:lstm_val_loss_vs_rmse_total} shows that most configurations cluster at low
test RMSE (approximately $0.165$--$0.195\,\mathrm{A}$) and low best validation loss
(approximately $0.07$--$0.18$), while a smaller set of configurations attains markedly higher
test RMSE values around $0.31$--$0.34\,\mathrm{A}$.

\begin{figure}[H]
    \centering
    \includegraphics[width=\linewidth]{Images/05_results/best_model_story/lstm/lstm_overfit_ratio_by_feature_mode.png}
    \caption{Overfit indicator (final validation loss / final training loss) by feature mode shown as boxplots across runs (green triangles indicate means, panels use different $y$-axis ranges for readability).}
    \label{fig:lstm_overfit_ratio}
\end{figure}

Figure~\ref{fig:lstm_overfit_ratio} reports the distribution of the final validation-to-training
loss ratio across feature modes.
Across the lower-range panel, the medians lie around $5$--$10$ depending on the mode, while
the high-range panel includes ratios extending beyond $30$.

\begin{figure}[H]
    \centering
    \includegraphics[width=\linewidth]{Images/05_results/best_model_story/lstm/lstm_residual_rmse_by_feature_mode.png}
    \caption{LSTM residual motor-current RMSE on the test split by feature mode shown as boxplots across runs (green triangles indicate means, panels use different $y$-axis ranges for readability).}
    \label{fig:lstm_residual_rmse_by_feature_mode}
\end{figure}

Figure~\ref{fig:lstm_residual_rmse_by_feature_mode} reports residual RMSE on the test split by
feature mode.
In the lower-range panel, the residual RMSE distributions are concentrated around
$0.17$--$0.19\,\mathrm{A}$, whereas the high-range panel contains configurations with residual
RMSE around $0.31$--$0.37\,\mathrm{A}$.

\begin{figure}[H]
    \centering
    \includegraphics[width=\linewidth]{Images/05_results/best_model_story/lstm/lstm_residual_rmse_by_H.png}
    \caption{LSTM residual motor-current RMSE on the test split shown as boxplots across history lengths $H$ (green triangles indicate means).}
    \label{fig:lstm_residual_rmse_by_window_length}
\end{figure}

Figure~\ref{fig:lstm_residual_rmse_by_window_length} shows residual RMSE grouped by history
length $H$.
Across $H\in\{50,100,150\}$, the medians lie in a similar range and the boxplots exhibit
comparable spread, with outliers present for all three history lengths.

\begin{figure}[H]
    \centering
    \includegraphics[width=\linewidth]{Images/05_results/best_model_story/lstm/residual_gt_vs_pred_test_H150.png}
    \caption{Representative residual motor-current traces: ground-truth residual versus LSTM residual prediction on the test split (valid indices $k\geq H-1$, $H=150$).}
    \label{fig:lstm_residual_gt_vs_predicted}
\end{figure}

Figure~\ref{fig:lstm_residual_gt_vs_predicted} provides a representative residual overlay for
all joints, showing the ground-truth residual and the corresponding LSTM prediction over time.

%%%%%%%%%%%%%%%%%%%%%%%%%%%%%%%%%%%%%%%%%%%%%%%%%%%%%%%%%%%%%%%%%%%%%%%%%%%%%%%%%%%%%%%%%%%%%%%%%%%%%%%%%%%%%%%%%%%%%%%%%%%%%%%%%%%%%%%%%%%%%%%%%%%%%%%%%%%%%%%%%%%%%%%%%%%%%%%%%%%%
\clearpage

Figure~\ref{fig:kdom_residual_overlay_example} reports the per-joint motor-current RMSE on the test
split for the selected DeLaN baseline and for the combined DeLaN+LSTM predictor (valid indices
$k\geq H-1$, $H=150$).
Across joints, the DeLaN bars range from approximately $0.12$ to $0.46\,\mathrm{A}$, whereas the
combined predictor ranges from approximately $0.07$ to $0.25\,\mathrm{A}$.
The ground-truth baseline is zero by construction and is therefore not visible as a bar.
The combined model attains lower RMSE than the DeLaN baseline on all joints, with the largest absolute reduction on joint~1 (the UR3 shoulder-lift joint), which is a major load-bearing joint of the arm.
The largest RMSE is observed on joint 1 and the smallest on joint 5 for both models.

\begin{figure}[H]
    \centering
    \includegraphics[width=\linewidth]{Images/05_results/best_model_story/evaluation/torque_rmse_per_joint_grouped_test_H150.png}
    \caption{Best pipeline motor-current RMSE per joint on the test split comparing DeLaN and the combined DeLaN+LSTM predictor (valid indices $k\geq H-1$, $H=150$).}
    \label{fig:kdom_residual_overlay_example}
\end{figure}

\clearpage

Figure~\ref{fig:kdom_torque_overlay_example} shows the corresponding motor-current time-series
overlay for the same evaluation, plotting the ground truth together with the DeLaN prediction and
the combined DeLaN+LSTM prediction for each joint.

\begin{figure}[H]
    \centering
    \includegraphics[width=\linewidth]{Images/05_results/best_model_story/evaluation/torque_gt_vs_delan_vs_combined_test_H150.png}
    \caption{Representative motor-current overlay: ground truth, DeLaN prediction, and combined DeLaN+LSTM prediction on the test split (valid indices $k\geq H-1$, $H=150$).}
    \label{fig:kdom_torque_overlay_example}
\end{figure}

\section{Performance Evaluation to Baseline}

To benchmark the proposed DeLaN+LSTM pipeline against prior work on the same dataset,
we compare against the data-driven dynamic model presented in~\cite{Q_Dataset_Paper}
and its accompanying IEEE DataPort release~\cite{Dataset_Dataport}.
The baseline study likewise applies low-pass filtering to the recorded signals as part of its preprocessing~\cite{Q_Dataset_Paper}.
The baseline identifies a joint-wise, linear-in-parameters model of the motor current $i_i$
from measured joint positions, velocities, and accelerations, and estimates the unknown
parameter vector via least squares:
\begin{equation}
  \mathbf{K}_i
  =
  \left(\boldsymbol{\Theta}_{m,i}^\top \boldsymbol{\Theta}_{m,i}\right)^{-1}
  \boldsymbol{\Theta}_{m,i}^\top \mathbf{i}_{m,i},
\end{equation}
where $\boldsymbol{\Theta}_{m,i}$ is the measurement matrix of the regressor terms for joint $i$
and $\mathbf{i}_{m,i}$ denotes the corresponding measured motor-current samples~\cite{Q_Dataset_Paper}.
Evaluation is reported as per-joint RMSE in current units,
\begin{equation}
  \mathrm{RMSE}(i_i)
  =
  \sqrt{\frac{1}{n}\sum_{k=1}^{n}\left(i_{i,k}^{\mathrm{real}}-i_{i,k}^{\mathrm{pred}}\right)^2}.
\end{equation}

Table~\ref{tab:baseline_vs_delan_fullpipeline_test}
reproduces the current-prediction RMSE reported by~\cite{Q_Dataset_Paper} for UR3e and UR10e,
each without load and with load, for the provided testing datasets.
In the baseline study, the model is fitted on $50{,}000$ training samples and evaluated on $5{,}000$ unseen test samples~\cite{Q_Dataset_Paper}.
Across conditions, UR3e baseline RMSE values are on the order of $0.06$--$0.38\,\mathrm{A}$,
while UR10e baseline RMSE values range up to $1.56\,\mathrm{A}$ on the test set.
The UR3e without-load DeLaN+LSTM model corresponds to the resulting best-model configuration selected in the Best Model Approach (Subsection~5.1.2).
The UR3e with-load and UR10e without-load DeLaN+LSTM models are trained using the same hyperparameter configuration identified in the Best Model Approach, and the corresponding benchmark test metrics are reported in the same table.

In addition to the dataset baseline, we report motor-current RMSE for the DeLaN and DeLaN+LSTM
pipeline under the same benchmark scale as the IEEE DataPort split, means training on $50{,}000$ samples
and testing on $5{,}000$ unseen samples.
The split is constructed at the trajectory level to prevent leakage of samples from training trajectories into
the test set.
Benchmark-level evaluation is reported for \texttt{UR3\_Load0Dataset}, \texttt{UR3\_Load2Dataset}, and
\texttt{UR10\_Load0Dataset}, which provide the full $50{,}000/5{,}000$ sample split.
To align the benchmark conditions with the IEEE DataPort baseline nomenclature, we treat
\texttt{UR3\_Load0Dataset} as the counterpart to ``UR3e without load'' and \texttt{UR10\_Load0Dataset} as the
counterpart to ``UR10e without load'' in Tables~\ref{tab:baseline_vs_delan_fullpipeline_test}.
Analogously, we benchmark \texttt{UR3\_Load2Dataset} against the baseline ``UR3e with load'' condition.
The remaining datasets (\texttt{UR10Load1Dataset}, \texttt{UR10Load2Dataset}, and \texttt{UR3Load1Dataset})
contain substantially fewer than $50{,}000$ total samples and are therefore not included in the benchmark-level
comparison.
Since the dataset provides motor current measurements, and motor current is treated as the primary
actuation signal throughout this thesis, we report RMSE in current units [A] per joint.
For the residual LSTM and the combined DeLaN+LSTM model, metrics are evaluated only on valid indices $k\geq H-1$ to account for the history window warm-up.

\begin{table}[H]
    \centering
    \scriptsize
    \setlength{\tabcolsep}{3pt}
    \caption{Baseline~\cite{Q_Dataset_Paper} vs. DeLaN vs. DeLaN+LSTM motor-current RMSE per joint on the benchmark test split (trained on $50{,}000$ samples, evaluated on $5{,}000$ unseen samples. DeLaN+LSTM evaluated on valid indices $k\geq H-1$).}
    \label{tab:baseline_vs_delan_fullpipeline_test}
    \resizebox{\linewidth}{!}{%
    \begin{tabular}{l l r r r r r r r r r r r r}
        \toprule
        Dataset & Model & $i_1$ [A] & \% & $i_2$ [A] & \% & $i_3$ [A] & \% & $i_4$ [A] & \% & $i_5$ [A] & \% & $i_6$ [A] & \% \\
        \midrule
        UR3e without load  & Baseline    & 0.089 & 3.14  & 0.130 & 2.87  & 0.122 & 4.55  & 0.066 & 4.39  & 0.106 & 8.29  & 0.083 & 5.89 \\
        UR3e without load  & DeLaN       & 0.247 & 0.764 & 0.464 & 0.555 & 0.195 & 0.386 & 0.147 & 0.525 & 0.287 & 0.969 & 0.124 & 0.982 \\
        UR3e without load  & DeLaN+LSTM  & 0.153 & 0.473 & 0.307 & 0.368 & 0.162 & 0.320 & 0.146 & 0.520 & 0.200 & 0.677 & 0.089 & 0.710 \\
        UR3e with   load   & Baseline    & 0.096 & 3.31  & 0.376 & 5.63  & 0.308 & 5.09  & 0.249 & 8.53  & 0.150 & 8.10  & 0.070 & 4.10 \\
        UR3e with   load   & DeLaN       & 0.295 & 0.696 & 0.833 & 0.523 & 0.442 & 0.466 & 0.244 & 0.356 & 0.231 & 0.601 & 0.180 & 0.600 \\
        UR3e with   load   & DeLaN+LSTM  & 0.052 & 0.122 & 0.142 & 0.089 & 0.098 & 0.103 & 0.080 & 0.116 & 0.051 & 0.133 & 0.043 & 0.144 \\
        UR10e without load & Baseline    & 0.490 & 3.24  & 0.853 & 2.54  & 0.351 & 3.14  & 0.110 & 4.40  & 0.098 & 6.48  & 0.078 & 5.12 \\
        UR10e without load & DeLaN       & 1.473 & 0.753 & 2.158 & 0.346 & 0.954 & 0.318 & 0.275 & 0.543 & 0.216 & 1.011 & 0.261 & 0.947 \\
        UR10e without load & DeLaN+LSTM  & 0.316 & 0.161 & 0.705 & 0.113 & 0.350 & 0.117 & 0.075 & 0.148 & 0.076 & 0.355 & 0.073 & 0.264 \\
        \bottomrule
    \end{tabular}%
    }
\end{table}

Table~\ref{tab:baseline_vs_delan_fullpipeline_test} reports per-joint motor-current RMSE on the
benchmark test split for the dataset baseline~\cite{Q_Dataset_Paper}, the learned DeLaN predictor,
and the combined DeLaN+LSTM model.
For UR3e without load, the baseline yields RMSE values between $0.066$ and $0.130\,\mathrm{A}$ across
joints, while DeLaN and DeLaN+LSTM range between $0.124$--$0.464\,\mathrm{A}$ and $0.089$--$0.307\,\mathrm{A}$, respectively.
For UR3e with load, DeLaN+LSTM reports lower RMSE than both the baseline and DeLaN on all joints, with values in the range $0.043$--$0.142\,\mathrm{A}$.
For UR10e without load, DeLaN exhibits substantially larger RMSE than the other models, whereas DeLaN+LSTM reports $0.073$--$0.705\,\mathrm{A}$ across joints.

\begin{figure}[H]
    \centering
    \includegraphics[width=\linewidth]{Images/05_results/benchmark/per_joint_relative_gain.png}
    \caption{Relative change in per-joint motor-current RMSE of DeLaN+LSTM versus the baseline model~\cite{Q_Dataset_Paper}, reported as $(\mathrm{RMSE}_{\mathrm{ours}}-\mathrm{RMSE}_{\mathrm{baseline}})/\mathrm{RMSE}_{\mathrm{baseline}}$ for UR3 (no load), UR3 (with load), and UR10 (no load). Negative values indicate a reduction in RMSE, positive values an increase.}
    \label{fig:benchmark_relative_gain_per_joint}
\end{figure}

Figure~\ref{fig:benchmark_relative_gain_per_joint} summarises the joint-wise relative change of
DeLaN+LSTM with respect to the baseline.
For UR3 (no load), the relative change is positive for all joints and spans approximately $+0.1$ to $+1.4$.
For UR3 (with load), the relative change is negative for all joints and lies approximately between $-0.4$ and $-0.7$.
For UR10 (no load), the relative change is negative across joints and ranges from near $0$ down to approximately $-0.35$.

\begin{figure}[H]
    \centering
    \includegraphics[width=\linewidth]{Images/05_results/benchmark/ur3_load_effect.png}
    \caption{Load sensitivity on UR3: overall motor-current RMSE for the baseline model~\cite{Q_Dataset_Paper} and DeLaN+LSTM, comparing the no-load and with-load conditions.}
    \label{fig:benchmark_ur3_load_sensitivity}
\end{figure}

Figure~\ref{fig:benchmark_ur3_load_sensitivity} reports overall motor-current RMSE on UR3 under
the no-load and with-load conditions.
The baseline increases from approximately $0.10\,\mathrm{A}$ (no load) to $0.21\,\mathrm{A}$ (with load),
whereas DeLaN+LSTM decreases from approximately $0.18\,\mathrm{A}$ to $0.08\,\mathrm{A}$.

\begin{figure}[H]
    \centering
    \includegraphics[width=\linewidth]{Images/05_results/benchmark/ur3_vs_ur10_no_load.png}
    \caption{Kinematic difference (UR3 versus UR10, no load): overall motor-current RMSE for the baseline model~\cite{Q_Dataset_Paper} and DeLaN+LSTM on the no-load test split for both manipulators.}
    \label{fig:benchmark_ur3_vs_ur10_no_load}
\end{figure}

Figure~\ref{fig:benchmark_ur3_vs_ur10_no_load} contrasts UR3 and UR10 on the no-load test split.
For the baseline model, the overall RMSE increases from approximately $0.10\,\mathrm{A}$ (UR3) to $0.33\,\mathrm{A}$ (UR10).
For the DeLaN+LSTM model, the overall RMSE increases from approximately $0.18\,\mathrm{A}$ (UR3) to $0.27\,\mathrm{A}$ (UR10).
