\chapter{State of the Art}
    \section{Research Strategy}
    The literature search was organised around five content clusters $C_1,\dots,C_5$ and the goal/context term sets $C_{mt}$ and $C_{ct}$. 
    The clusters capture the main methodological families, while $C_{mt}$ and $C_{ct}$ constrain the queries to estimation-related objectives in robotic manipulation:

    \begin{itemize}
    \item $C_1  =$ Classical / Observers
    \item $C_2  =$ Gaussian Process (GP)
    \item $C_3  =$ Deep Sequence Models (MLP / GRU / TCN / Transformer / LSTM)
    \item $C_4  =$ Physics-Informed / Differentiable
    \item $C_5  =$ Surveys
    \item $C_T  =$ Goal \& Domain Terms
    \begin{itemize}
        \item $C_{mt} =$ Estimation \& Modeling Terms
        \item $C_{ct} =$ Robotics Context Terms
    \end{itemize}
    \end{itemize}

    The detailed index terms associated with each set are listed in 
    Appendix~\ref{app:query_categories}. For each content cluster $C_i$, a family of queries $Q_i$ was constructed by combining
    (disjunctions of) its index terms with estimation \& modelling terms from $C_{mt}$ and robotics context terms from $C_{ct}$.
    Figure~\ref{fig:query_logic} illustrates this logic schematically as a generalised set intersection over the three term groups.

    \begin{figure}[H]
    \centering
    \includegraphics[width=1\columnwidth]{Images/2_state_of_art/query_logic.drawio.png}
    \caption[Query logic]{Query logic used to categorise the SoA papers. 
        Each category $Q_i$ is formed by combining content clusters $C_i$ with estimation \& modelling terms $C_{mt}$ and robotics context terms $C_{ct}$. 
        The combined representation $C$ and query set $Q$ are formed by the union of their respective subsets.}
    ~\label{fig:query_logic}
    \end{figure}

    \vspace{0.75\baselineskip}

    This process yielded 36 papers that are directly relevant to robot and payload dynamics, interaction force estimation and related identification problems.
    Table~\ref{tab:query-results} summarises how these works are distributed across the five query categories and counts, for each category, how many papers address payload dynamics, robot rigid-body dynamics,
    and contact-force estimation (papers that treat payload and rigid-body dynamics contribute to both columns). The corresponding relations are visualised in the concept graph in Fig.~\ref{fig:SoA_concept_graph}.

    \begin{table}[H]
    \centering
    \caption{Overview of query results by category.}
    ~\label{tab:query-results}
    \begin{tabular}{lcccc}
        \toprule
        Query & Relev.\ SoA & Payload & Rigid-body & Contact Force \\
        \midrule
        $\mathbf{Q_1 =}$ Classical / Observers        & 17 & 10 & 8 & 3  \\
        $\mathbf{Q_2 =}$ Gaussian Process (GP)        &  4 & 0  & 1  & 3  \\
        $\mathbf{Q_3 =}$ Deep Sequence Models         &  8 & 4  & 3  & 1  \\
        $\mathbf{Q_4 =}$ Physics-Informed / Diff.\    &  5 & 0  & 5  & 0  \\
        $\mathbf{Q_5 =}$ Surveys                      &  2 & -- & -- & -- \\ % chktex 8
        \midrule
        \textbf{Total}                                & \textbf{36} & \textbf{14} & \textbf{17} & \textbf{7} \\
        \bottomrule
    \end{tabular}
    \end{table}

    \vspace{0.75\baselineskip}

\section{Literature}\label{sec:literature}
In the context of robotic manipulator dynamics and payload identification, existing methods largely fall into the four methodological families reproduced by Lutter et al.~\cite{Q4_2_lutter2023combiningphysicsdeeplearning} and
illustrated in Figure~\ref{fig:approach_landscape}.

\vspace{0.75\baselineskip}

\begin{figure}[H]
  \centering
  \includegraphics[width=\linewidth]{Images/2_state_of_art/Q4_2_Method_overview_grafic.png}
  \caption[Overview of modelling paradigms]{Overview of modelling paradigms for robot dynamics
  (reproduced by Lutter et al.~\cite{Q4_2_lutter2023combiningphysicsdeeplearning}). The four panels correspond
  to the approaches discussed in this SoA: classical rigid-body model engineering and system
  identification (Category~Q1), physics-inspired networks such as DeLaN/PINNs (Category~Q4), and
  black-box model learning with deep networks (Category~Q3). Gaussian-process residual models
  (Category~Q2) sit between system identification and black-box learning.}
  ~\label{fig:approach_landscape}
\end{figure}

\clearpage

The Category Q1 `Classical / Observers' groups classical \emph{model-based methods for robot and payload dynamics and interaction force estimation}, mostly based on linearly parameterised rigid-body dynamics (RBD) and LS/WLS-type Newton-Euler regressors,
sometimes combined with observers and Kalman filters~\cite{Q1_1_fast_inertial_id_cobots,Q1_2_online_payload_mo,Q1_3_external_torque_smo,Q1_4_contact_force_kf,Q1_5_payload_estimation_compensation_2025,Q1_6_hu2025_fdrdi,Q1_7_10947736,Q1_8_10944553,Q1_9_xu2022_double_weighting_payload_id,Q1_10_duan_payload_ftsensor,Q1_12_han_finite_time_observer,Q1_13_dynamic_model_id_swevers2007,Q1_14_long2022sliding_momentum_observer,Q1_15_tang2023wls_rwspo,Q1_16_xu_payload_difference_2024,Q1_17_liu2021sensorless_dob_nn}.
Across these works, three main lines of research can be distinguished in payload dynamic parameter identification (PDPI) using force/torque (FT) sensing~\cite{Q1_1_fast_inertial_id_cobots,Q1_5_payload_estimation_compensation_2025,Q1_7_10947736,Q1_10_duan_payload_ftsensor},
robot and payload dynamic parameter identification in joint or motor-current space without FT sensors~\cite{Q1_6_hu2025_fdrdi,Q1_8_10944553,Q1_9_xu2022_double_weighting_payload_id,Q1_13_dynamic_model_id_swevers2007,Q1_15_tang2023wls_rwspo,Q1_16_xu_payload_difference_2024},
and observer-based sensorless force/torque estimation and online payload identification using proprioceptive data~\cite{Q1_3_external_torque_smo,Q1_4_contact_force_kf,Q1_12_han_finite_time_observer,Q1_14_long2022sliding_momentum_observer,Q1_17_liu2021sensorless_dob_nn,Q1_2_online_payload_mo}.  
Across Q1, mass is usually identified accurately, CoM moderately well, and inertia emerges as the hardest quantity to estimate robustly~\cite{Q1_1_fast_inertial_id_cobots,Q1_5_payload_estimation_compensation_2025,Q1_7_10947736,Q1_10_duan_payload_ftsensor,Q1_16_xu_payload_difference_2024}.

A first group of methods performs PDPI directly in the FT frame~\cite{Q1_1_fast_inertial_id_cobots,Q1_5_payload_estimation_compensation_2025,Q1_7_10947736,Q1_10_duan_payload_ftsensor}.
They typically exploit static poses to identify the payload mass and centre of mass, and then use dedicated dynamic excitation trajectories together with LS or TLS-type Newton-Euler regressors to estimate the inertia tensor.
Representative works demonstrate that, given a sufficiently informative excitation and an FT sensor rigidly mounted at the flange, payload mass can be recovered very accurately and CoM can be estimated with reasonable
precision, even for heavy payloads~\cite{Q1_1_fast_inertial_id_cobots,Q1_7_10947736}.  
However, inertia estimates are systematically more fragile, especially under cobot-typical safety constraints with short trajectories and low accelerations and in several cases quantitative ground truth for CoM and inertia
is missing or only partially available (validation is often given in terms of residual gravitational/inertial wrench after compensation rather than direct parameter error)~\cite{Q1_1_fast_inertial_id_cobots,Q1_5_payload_estimation_compensation_2025,Q1_7_10947736,Q1_10_duan_payload_ftsensor}.  
Moreover, these approaches require additional FT hardware and considerable experimental effort in the form of carefully designed calibration motions.

A second group tackles robot dynamic parameter identification (RDPI) and PDPI in joint space or motor-current space without FT sensors~\cite{Q1_6_hu2025_fdrdi,Q1_8_10944553,Q1_9_xu2022_double_weighting_payload_id,Q1_13_dynamic_model_id_swevers2007,Q1_15_tang2023wls_rwspo,Q1_16_xu_payload_difference_2024}.  
Here, fully or partially decoupled identification schemes are designed to separate gravitational, frictional and inertial effects, often using families of S-curve or Fourier trajectories executed both with and without
payload~\cite{Q1_6_hu2025_fdrdi,Q1_8_10944553,Q1_9_xu2022_double_weighting_payload_id,Q1_13_dynamic_model_id_swevers2007}.  
Double-weighted WLS and optimisation-enhanced LS methods achieve very accurate joint-torque prediction and good agreement with CAD-based payload models, confirming that classical LS/NE pipelines, for example by Swevers et al.~\cite{Q1_13_dynamic_model_id_swevers2007} remain a strong baseline
for RDPI and PDPI~\cite{Q1_6_hu2025_fdrdi,Q1_8_10944553,Q1_9_xu2022_double_weighting_payload_id,Q1_13_dynamic_model_id_swevers2007,Q1_15_tang2023wls_rwspo,Q1_16_xu_payload_difference_2024}.
At the same time, these methods are typically executed in dedicated calibration phases, rely on repeated execution of long and highly exciting trajectories and payload parameters are updated only between such identification
runs. They therefore provide an excellent commissioning tool and basis model, but do not by themselves endow the robot with continuous online awareness of payload changes during manipulation tasks.

A third line of work focuses on sensorless estimation of external joint torques and end-effector wrenches using observers and filters~\cite{Q1_3_external_torque_smo,Q1_4_contact_force_kf,Q1_12_han_finite_time_observer,Q1_14_long2022sliding_momentum_observer,Q1_17_liu2021sensorless_dob_nn}.
Momentum observers, higher-order sliding-mode observers, adaptive Kalman filters and high-order finite-time observers use a nominal RBD model together with controller torques and joint measurements to reconstruct external forces,
sometimes with probabilistic covariance information.  
These methods achieve good performance in collision detection, binary contact decisions and execution monitoring, and some approaches augment classical friction models with learned nonlinear terms such as neural-network
Stribeck approximations~\cite{Q1_3_external_torque_smo,Q1_14_long2022sliding_momentum_observer,Q1_17_liu2021sensorless_dob_nn}.
Nevertheless, their accuracy depends critically on the quality of the underlying RBD and friction models, and residual force errors remain significant in highly dynamic phases or around velocity reversals~\cite{Q1_3_external_torque_smo,Q1_12_han_finite_time_observer,Q1_17_liu2021sensorless_dob_nn}.
Importantly, most observer-based schemes treat payloads and tools as fixed parts of the nominal model or as lumped disturbances, and do not explicitly estimate payload parameters.

More recent contributions bridge RDPI/PDPI and observer-based estimation by using proprioceptive data to identify payload parameters online~\cite{Q1_2_online_payload_mo,Q1_16_xu_payload_difference_2024}.
Momentum-observer-based schemes and parameter-difference methods compute external joint torques as residuals between measured and model-based torques and apply LS/RLS Newton-Euler regressors to recover payload mass, CoM and,
in some cases, inertia during regular robot operation.  
These works demonstrate that accurate online PDPI is possible without FT sensors, provided that a reasonably accurate base robot model, friction compensation and sufficiently exciting motions are available~\cite{Q1_2_online_payload_mo,Q1_16_xu_payload_difference_2024}.
At the same time, they underline several structural limitations: inertia remains the hardest quantity to identify robustly. Nonlinear friction, backlash and transmission effects must be modelled or learned carefully (with several authors explicitly noting residual error peaks near motion reversal due to unmodelled friction~\cite{Q1_13_dynamic_model_id_swevers2007,Q1_15_tang2023wls_rwspo}).
and the resulting estimators not providing a unified, continuously updated representation of robot and load.

%%%%%%%%%%%%%%%%%%%%%%%%%%%%%%%%%%%%%%%%%%%%%%%
\vspace{0.75\baselineskip}

The Category Q2 `Gaussian Process (GP)' groups \emph{Gaussian Process (GP) and GP-hybrid methods for inverse dynamics and sensorless contact estimation}. 
In all cases, a GP is trained offline as a residual or surrogate model on top of a nominal rigid-body dynamics (RBD) description, and then deployed online for torque or disturbance prediction
~\cite{Q2_1_contact_force_gp_observer,Q2_3_contact_detection_gp}

Two studies use GPs to improve joint-space contact-force estimation.
In~\cite{Q2_1_contact_force_gp_observer}, an enhanced GP learns the residual dynamics between an Euler-Lagrange model and measured torques. This residual is injected into a decoupling disturbance observer and Kalman filter
to obtain external joint torques and end-effector forces.  
The follow-up~\cite{Q2_2_contact_force_gpadkf} extends this to a GP-adaptive disturbance Kalman filter, where the disturbance covariance is adjusted based on the GP output.
Both papers show reduced estimation error and faster convergence than purely model-based observers, but require a reasonably accurate nominal model, high-quality proprioception and extensive non-contact training data.
Payload and tool effects are absorbed into a single residual term.

A third work combines GP inverse dynamics with learning-based contact detection~\cite{Q2_3_contact_detection_gp}.
A GP predicts non-contact motor torques from joint states, and the residual between measured and GP-predicted torques is passed to a convolutional neural network that classifies contact vs.\ no-contact during assembly tasks,
achieving high classification accuracy on scripted collisions.  
The method remains task-specific and does not recover physical interaction forces or payload parameters.

Finally,~\cite{Q2_4_GIACOMUZZO20231584} compares GPs and neural networks for inverse dynamics when the inputs include physics-inspired features (e.g.\ nominal RBD torques).  
Embedding such structure improves data efficiency and prediction accuracy and GPs are competitive for moderate dataset sizes.

%%%%%%%%%%%%%%%%%%%%%%%%%%%%%%%%%%%%%%%%%%%%%%%
\vspace{0.75\baselineskip}

\clearpage

Deep Sequence Models comprises deep-learning approaches for inverse dynamics, force estimation and payload identification.  
Most methods either learn residual dynamics on top of a nominal rigid-body model or learn a direct mapping from joint histories to payload parameters or contact indicators, using LSTM/GRU-type sequence models, feed-forward
networks, CNNs or ensemble methods~\cite{Q3_1_tao_bll,Q3_2_encoder_attention_payload,Q3_3_lstm_force_estimation,Q3_4_asgrnn_force_observer,Q3_5_contact_localization_cnn,Q3_6_payload_id_incremental_ensemble,Q3_7_payload_id_online_ensemble,Q3_8_payload_id_catastrophic_forgetting}.

A first group focuses on \emph{deep residual inverse dynamics} on top of a nominal RBD model.  
In~\cite{Q3_1_tao_bll}, the authors use the public Franka Panda dataset and the Gaz et al.\ model to compute joint-torque residuals between data and RBD prediction, and train a bootstrapped
LSTM ensemble (BLL-LSTM) on sequences of $(q,\dot q,\ddot q)$ to predict these residuals.  
The ensemble clearly improves torque prediction over Gaussian processes and single models on held-out dataset splits, but remains a purely offline, dataset-based study that assumes a reasonably accurate full robot model.  
Similarly,~\cite{Q3_3_lstm_force_estimation} trains LSTMs to map base FT-wrench and joint states to end-effector tip forces, and joint states to joint torques, on a 7-DoF Panda.  
The LSTMs outperform MLPs, 1D convolutions and a DeLaN baseline in both simulation and real experiments, but are evaluated in a task-specific setting and rely on FT hardware and simulation-generated ground-truth forces.

A second group uses deep models to infer end-effector wrench or contact directly from proprioception.  
The adaptive sparse GRNN force observer in~\cite{Q3_4_asgrnn_force_observer} maps joint positions, velocities, accelerations and motor currents to 6D wrench on a UR5 teleoperation system, using an FT sensor only for
supervision. It achieves strong soft/stiff collision force estimation and outperforms GP and MLP baselines.  
In~\cite{Q3_5_contact_localization_cnn}, a CNN is trained in IsaacGym with domain randomisation to detect and localise link-environment contacts for a 7-DoF Panda using only link velocities and pose errors
(no torque or FT sensing). The network reaches around 98\,\% accuracy in sim-to-real contact localisation, but does not estimate wrench magnitude or payload dynamics.

The third line of work addresses \emph{payload dynamic parameter identification (PDPI)} with deep networks and ensembles.  
A learning-based method for the OpenMANIPULATOR-X in~\cite{Q3_2_encoder_attention_payload} combines a nominal RBD model and camera pose estimation with an MLP that processes joint states and velocity sign to estimate
per-joint torque contributions. A subsequent LS step recovers payload mass and CoM of known objects, with average errors of about 9\,\% in mass and 18\,\% in CoM.  
For collaborative cobots,\cite{Q3_7_payload_id_online_ensemble} proposes a batch ensemble of weak learners (NNs or decision trees) that directly map $(q,\dot q,\tau)$ to payload parameters for a library of 77 payloads along
a fixed excitation trajectory on a Franka robot. The method achieves good sim-to-real transfer and clearly reduces mass/CoM error compared to RLS, but still requires a dedicated excitation path per payload.  
This is generalised in~\cite{Q3_6_payload_id_incremental_ensemble,Q3_8_payload_id_catastrophic_forgetting}, which introduce incremental ensemble learning for PDPI along arbitrary task paths.  
The initial incremental ensemble~\cite{Q3_6_payload_id_incremental_ensemble} adapts weak learners online based on Euclidean distance in feature space but suffers from catastrophic forgetting on previously seen trajectories.  
The follow-up work~\cite{Q3_8_payload_id_catastrophic_forgetting} adds a bag-based classifier that routes new path segments to the most relevant weak learner and spawns new learners as required, thereby preserving accuracy
on old paths while adapting to new ones and eliminating the explicit excitation trajectory.  
The price is a growing ensemble size and the fact that the individual learners remain small feed-forward networks without explicit temporal structure.

%%%%%%%%%%%%%%%%%%%%%%%%%%%%%%%%%%%%%%%%%%%%%%%%%%%
\vspace{0.75\baselineskip}

Physics-informed and differentiable dynamics models group Category Q4, where deep networks are structured by Lagrangian or state-space physics and trained on joint data to predict torques or currents.
The central goal in all works is accurate robot dynamic parameter identification (RDPI) and inverse dynamics prediction for fixed robots. Payload dynamics and interaction forces are not estimated explicitly.

A first line of work builds on Deep Lagrangian Networks (DeLaN) and related continuous-time models. The survey and benchmark in~\cite{Q4_2_lutter2023combiningphysicsdeeplearning} compares structured DeLaN/HNN models against
black-box neural networks on low-DoF systems and a 4-DoF WAM arm, showing that physics-informed architectures achieve lower normalised errors and much longer valid prediction horizons than unconstrained networks. However,
these models assume conservative dynamics without contacts, and friction is either neglected or handled separately. Extending this idea,~\cite{Q4_1_extended_delan_motor} learns an ``extended DeLaN'' that incorporates motor
couplings and friction on a UR10e arm, using joint positions, velocities and accelerations together with motor currents. The network learns Lagrangian terms plus actuator/friction parameters and predicts motor currents with
high accuracy in simulation and on hardware, outperforming the original DeLaN and a feed-forward baseline, yet still under fixed tool/payload and contact-free conditions.

A second line of work uses physics-informed neural networks (PINNs) as refinements of classical LS/NE identification. In~\cite{Q4_3_residual_pinns_dynamics_id}, base parameters are first obtained by a Newton-Euler regressor.
A decomposed PINN then minimises a hybrid loss combining data residuals and the rigid-body dynamics equations, reducing joint-torque RMSE compared to LS alone. Building on this,~\cite{Q4_4_10729277}
proposes a friction-inclusive PINN for multi-joint industrial robots without joint torque sensors, combining Lagrangian dynamics with an explicit Stribeck friction model and a history-based residual network that learns
remaining errors over a time window. The resulting hybrid model achieves very strong joint-torque (or current) prediction across several joints and outperforms DeLaN-type and LS baselines. Finally,~\cite{Q4_5_10305255}
introduces an H-PINN for a single collaborative robot joint, embedding the joint's state-space dynamics into an RNN cell and jointly estimating physical parameters and state transitions, with highly accurate joint-level
dynamics prediction in simulation and experiments.

%%%%%%%%%%%%%%%%%%%%%%%%%%%%%%%%%%%%%%%%%%%%%%%%%%%%%
\vspace{0.75\baselineskip}

Taken together, model-based methods can deliver high-quality RDPI and PDPI, as well as useful sensorless interaction-force estimates, but typically only under carefully
controlled excitation of dedicated identification trajectories~\cite{Q1_1_fast_inertial_id_cobots,Q1_5_payload_estimation_compensation_2025,Q1_6_hu2025_fdrdi,Q1_7_10947736,Q1_8_10944553,Q1_9_xu2022_double_weighting_payload_id,Q1_12_han_finite_time_observer,Q1_13_dynamic_model_id_swevers2007,Q1_15_tang2023wls_rwspo,Q1_16_xu_payload_difference_2024,Q1_17_liu2021sensorless_dob_nn}.
From the perspective of this work, the main gaps are the lack of a unified, online notion of dynamic awareness that covers robot and payload; the persistent difficulty of reliably identifying and exploiting
payload inertia in cobot-safe regimes; and the sensitivity of existing approaches to friction and transmission nonlinearities~\cite{Q1_2_online_payload_mo,Q1_3_external_torque_smo}.

Gaussian Processes show up as effective residual models for unmodelled robot dynamics and can enhance sensorless contact estimation
~\cite{Q2_1_contact_force_gp_observer,Q2_2_contact_force_gpadkf,Q2_3_contact_detection_gp,Q2_4_GIACOMUZZO20231584}.
However, the GP is always a lumped compensator: there is no sequence-aware handling of backlash or history-dependent friction,
and no unified dynamic representation in the measurement frame that could serve as a precise, task-agnostic basis for tool/gripper compensation.

\clearpage

The literature shows that deep models can substantially improve torque and force estimation and can achieve competitive PDPI, including sim-to-real transfer for collaborative robots.  
At the same time, existing works either depend on accurate nominal models and FT supervision, or operate as largely black-box payload regressors, and none provide a unified deep sequence model that simultaneously
delivers joint-torque prediction, tool/payload compensation and interaction-force awareness during general manipulation tasks.

Across Physics-Informed and Differential Models, physics-informed deep models consistently improve inverse-dynamics prediction and friction compensation compared to purely black-box networks, while using only encoder and motor data~\cite{Q4_1_extended_delan_motor,Q4_2_lutter2023combiningphysicsdeeplearning,Q4_3_residual_pinns_dynamics_id,Q4_4_10729277,Q4_5_10305255}.
At the same time, they are trained on carefully designed excitation trajectories and then deployed as fixed models, without explicit treatment of changing payloads or contact forces, and some architectures become quite
complex when scaling beyond low-DoF setups or single joints.
Thus, physical informed and differnetial models provides strong structured baselines for fixed robot dynamics, but does not yet realise an online, unified dynamic awareness of robot, tool and payload with explicit force estimation, as targeted in this work.

Overall, the literature spans the full spectrum in Fig.~\ref{fig:approach_landscape}: from classical model engineering and system identification, through GP residual models, to black-box deep learning and
physics-inspired networks. Classical rigid body dynamic observer and model based methods provide strong RDPI/PDPI under carefully designed calibration trajectories, but remain sensitive to friction and payload changes and rarely yield a
unified, online notion of the effective rigid body. Gaussian process and deep black-box models are robost as residual models to cover unmodelled dynamics and contact, yet largely treat friction, payload and interaction forces as a single residual and
do not produce a physics-consistent wrench model in the measurement frame. Physics-informed DeLaN/PINN approaches in the from the fourth category bridge these extremes by embedding Lagrangian structure and achieving joint-torque
prediction from proprioception alone, but are typically trained for fixed tools and evaluated only in joint space. This thesis therefore positions itself in the “physics-inspired networks” quadrant of
Fig.~\ref{fig:approach_landscape}, combining a DeLaN-style backbone with an LSTM residual model to obtain a single joint-space inverse-dynamics model that also serves as a reliable nominal wrench model in the measurement
frame and a basis for PDPI.

\section{Limitations of the Current State of the Art}
The reviewed literature reveals several recurring limitations that motivate the approach developed in this thesis:

\begin{itemize}
\item \textbf{Calibration-centric identification and limited continuity across tasks.}
Accurate RDPI/PDPI results are commonly obtained in dedicated calibration phases using carefully designed excitation trajectories and repeated with/without-payload experiments~\cite{Q1_1_fast_inertial_id_cobots,Q1_6_hu2025_fdrdi,Q1_7_10947736,Q1_8_10944553,Q1_9_xu2022_double_weighting_payload_id,Q1_13_dynamic_model_id_swevers2007,Q1_16_xu_payload_difference_2024,Q3_2_encoder_attention_payload,Q3_6_payload_id_incremental_ensemble,Q3_7_payload_id_online_ensemble,Q3_8_payload_id_catastrophic_forgetting}.
Even online variants presuppose a sufficiently accurate nominal model and friction compensation~\cite{Q1_2_online_payload_mo,Q1_16_xu_payload_difference_2024}.
This leaves a gap for methods that provide a single, continuously usable nominal dynamics model during general manipulation tasks.

\item \textbf{Incomplete treatment of payload inertia and the effective rigid body.}
In Q1, payload mass (and often CoM) can be estimated accurately, but payload inertia is frequently weakly excited, ill-conditioned, or only partially validated, especially in cobot-safe regimes~\cite{Q1_1_fast_inertial_id_cobots,Q1_5_payload_estimation_compensation_2025,Q1_7_10947736,Q1_10_duan_payload_ftsensor,Q1_16_xu_payload_difference_2024}.
Learning-based approaches in Categories~Q2--Q4 largely assume fixed tools/payloads and therefore do not maintain an explicit, continuously updated effective rigid-body representation~\cite{Q2_1_contact_force_gp_observer,Q2_4_GIACOMUZZO20231584,Q3_3_lstm_force_estimation,Q3_4_asgrnn_force_observer,Q4_1_extended_delan_motor,Q4_3_residual_pinns_dynamics_id,Q4_4_10729277}.

\item \textbf{Entangled treatment of friction, transmission, and contact effects.}
Observer-based schemes and LS/NE identification remain sensitive to imperfect friction and transmission models. Residual errors often peak around velocity reversals and during highly dynamic motion phases~\cite{Q1_3_external_torque_smo,Q1_4_contact_force_kf,Q1_6_hu2025_fdrdi,Q1_12_han_finite_time_observer,Q1_15_tang2023wls_rwspo,Q1_17_liu2021sensorless_dob_nn}.
GP and deep residual models can reduce these errors but commonly absorb friction, backlash, and contact effects into a single disturbance term~\cite{Q2_1_contact_force_gp_observer,Q2_2_contact_force_gpadkf,Q2_3_contact_detection_gp,Q3_1_tao_bll,Q3_3_lstm_force_estimation,Q4_4_10729277}, which complicates separating payload-induced dynamics from interaction-induced effects in a physics-consistent way.

\item \textbf{Limited integration of temporal residual learning with physics-structured models in the actuation domain.}
Deep sequence models are frequently used either as black-box regressors from motion histories to torques/forces or as residual predictors on top of rigid-body baselines~\cite{Q3_1_tao_bll,Q3_3_lstm_force_estimation,Q3_4_asgrnn_force_observer,Q3_5_contact_localization_cnn}.
Physics-informed DeLaN/PINN models improve inverse-dynamics prediction and can leverage motor-side measurements when joint-torque sensing is unavailable~\cite{Q4_1_extended_delan_motor,Q4_2_lutter2023combiningphysicsdeeplearning,Q4_4_10729277,Q4_6_HU2026103093}, but are typically trained and evaluated as fixed predictors.
This motivates a pragmatic two-stage architecture that combines a physics-consistent inverse-dynamics backbone with a learned history-dependent residual, trained from encoder and motor-current data, to serve as a nominal model for downstream tool/payload compensation and PDPI.
\end{itemize}

%%%%%%%%%%%%%%%%%%%%%%%%%%%%%%%%%%%%%%%%%%%%%%%%%%%%%

    \begin{figure}[H]
    \centering
    \includegraphics[width=1\columnwidth]{Images/2_state_of_art/251202_concept_graph.drawio.png}
    \caption[Concept graph of the SoA literature]{The Concept graph summarising the reviewed literature. 
    Green nodes represent method categories Q1-Q4 with node the size proportional to the number of papers in each category. 
    Blue nodes denote the main estimation goals scaled by how many papers address each goal. 
    Red nodes correspond to individual references R1-R34. Their diameter is proportional to the citation count (clipped at the maximum). 
    Directed edges indicate that a given reference belongs to a method category and contributes to one or more estimation goals. R1-R34 table in Appendix \ref{soa_references_graph}}

    ~\label{fig:SoA_concept_graph}
    \end{figure}

\clearpage

\section{Related Work}~\label{subsec:relatedWork}

This thesis is most closely related to physics-informed inverse-dynamics models that combine Lagrangian structure with learning-based compensation in settings without direct joint-torque supervision.
Wu et al.~\cite{Q4_1_extended_delan_motor} propose an extended DeLaN formulation that incorporates motor-side effects and couplings and learns dynamics directly from electrical motor signals. This motivates current-domain supervision when only motor currents are available.
Hu et al.~\cite{Q4_4_10729277} develop a friction-inclusive, PINN-style dynamics model that combines a structured backbone with an explicit friction model and a sequential residual learner (TCN), which is conceptually closest to the two-stage ``structured baseline + temporal residual'' design adopted in this thesis.
Finally, the DeLaN framework and benchmark presented by Lutter et al.~\cite{Q4_2_lutter2023combiningphysicsdeeplearning} provides the methodological baseline for physics-inspired dynamics learning and serves as the code-level starting point for the DeLaN implementation used in this work.

%%%%%%%%%%%%%%%%%%%%%%%%%%%%%%%%%%%%%%%%%%%%%%%%%%%%%%%%%%

\section{Deep Lagrangian Networks}
Deep Lagrangian Networks (DeLaN) are physics-inspired neural models for rigid-body dynamics that embed Lagrangian mechanics directly into the network structure.
Instead of learning a black-box mapping from state to actuation, DeLaN parameterises the system energy and derives the dynamics via the Euler--Lagrange equations, which yields models that are physically plausible by construction and provide access to multiple consistent quantities (energy, forward dynamics, inverse dynamics) from the same set of parameters~\cite{Q4_2_lutter2023combiningphysicsdeeplearning}.
Concretely, DeLaN represents the Lagrangian $\mathcal{L}(\mathbf{q},\dot{\mathbf{q}})=T(\mathbf{q},\dot{\mathbf{q}})-V(\mathbf{q})$ by learning a configuration-dependent inertia matrix and a potential-energy term, and then evaluates the implied inverse dynamics
\(
  \hat{\boldsymbol{\tau}}(\mathbf{q},\dot{\mathbf{q}},\ddot{\mathbf{q}})
\)
through analytic differentiation.
To respect key structural properties, common DeLaN implementations enforce symmetry and positive definiteness of the learned inertia (e.g., via a Cholesky-factor parameterisation), while non-conservative effects are modelled through an additional friction term or residual component~\cite{Q4_2_lutter2023combiningphysicsdeeplearning,Q4_4_10729277}.
DeLaN models are well suited to dynamics-learning problems that are strongly constrained by mechanics.

\section{Long-Short-Term-Memory}~\label{sec:lstm_general}

Long short-term memory (LSTM) networks are recurrent neural networks designed to learn temporal dependencies in sequential data.
An LSTM processes a time series step by step and maintains an internal cell state as a dedicated memory, whose content is controlled by multiplicative gates (forget, input, output).
This gating mechanism enables the model to selectively retain or overwrite information over long horizons, which is crucial when dynamics depend on history rather than solely on the instantaneous state.

In robotics, LSTMs are therefore frequently used as sequence-to-vector or sequence-to-sequence regressors to infer forces and torques from histories of proprioceptive signals.
Kru\v{z}i\'{c} et al.~\cite{Q3_3_lstm_force_estimation} demonstrate this idea for a 7-DoF manipulator by training deep networks (including an LSTM) to estimate end-effector forces and joint torques from measured motion signals, without requiring an explicit analytical dynamics model.
Tao et al.~\cite{Q3_1_tao_bll} apply an LSTM as a residual predictor in a hybrid inverse-dynamics compensation scheme, where the recurrent model learns the unmodelled error terms (e.g., due to friction and other uncertainties) on top of a rigid-body baseline and thereby reduces the remaining torque residuals.
