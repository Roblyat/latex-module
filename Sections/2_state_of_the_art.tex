\chapter{State of the Art}
\section{One-page condensed SoA summary for Q1}\label{sec:Q1_SoA_summary}
Category Q1 groups classical \textbf{model-based methods for robot and payload dynamics and interaction force estimation}, mostly based on \textbf{linearly parameterised rigid-body dynamics (RBD) and LS/WLS regressors}, sometimes combined with observers and Kalman filters.
TTTTTEEEEESSSSSSSTTTTTTTT
Across the papers\cite{Q1_7_10947736}\cite{Q1_8_10944553}\cite{Q1_9_xu2022_double_weighting_payload_id}, \textbf{Least Squares (LS),
    Weighted LS (WLS) and LS-Newton-Euler (LS-NE) regressors} are the dominant
tools for both \textbf{robot dynamic parameter identification (RDPI)} and
\textbf{payload dynamic parameter identification (PDPI)}. They are used in
joint space, in motor-current space and in sensor frames, and provide
\textbf{strong performance for mass, centre of mass (CoM) and joint-torque
    prediction} when trajectories are sufficiently exciting.

A large subset of works demonstrates that \textbf{neither a nominal CAD-based
    RBD model nor an FT sensor is strictly necessary}.\cite{Q1_1_fast_inertial_id_cobots},\cite{Q1_15_tang2023wls_rwspo},\cite{Q1_5_payload_estimation_compensation_2025},\cite{Q1_6_hu2025_fdrdi},\cite{Q1_7_10947736}, Q1.5--Q1.8, Q1.13, Q1.15 and Q1.16 build the regressor
directly from measured joint states and controller torques, sometimes in
\textbf{fully decoupled formulations} or via \textbf{residual-torque
    decomposition}. They use \textbf{constant-velocity/acceleration S-curve
    trajectories, Fourier trajectories or repeated sections} to decorrelate
parameters and improve conditioning. These approaches typically obtain
\textbf{very good mass estimates and acceptable CoM}, with \textbf{inertia
    remaining the weakest part of the identification}, especially for short
trajectories.

Several methods employ \textbf{two-stage pipelines}: static poses for mass and
CoM, followed by dynamic trajectories for inertia (e.g.\ Q1.5 and Q1.7). Q1.7
also shows that such schemes scale to \textbf{heavy ($\sim 40\,\mathrm{kg}$)
    payloads}, and can feed into \textbf{contact force estimation and compensation}
with moderate batch times ($\approx 10\,\mathrm{s}$ for contact, $\approx
    40\,\mathrm{s}$ for payload).

Where an \textbf{NRB model is available or identified offline}, it is commonly
combined with \textbf{observers} for external torque and force estimation. Q1.2
and Q1.3 use LS-NE-based torque prediction together with \textbf{momentum or
    sliding-mode observers} to estimate external joint torques and EE forces. Q1.4
and Q1.17 combine RBD with \textbf{(adaptive) Kalman filters / disturbance
    observers} and explicit friction models (Stribeck or NN-based). These
approaches can yield \textbf{good EE force estimation and collision detection},
but they are sensitive to model mismatch and friction modelling; Q1.17 reports
good behaviour without external forces but large errors (up to $\approx
    9\,\mathrm{Nm}$) under contact. \textbf{Sensorless interaction-force
    estimation} is addressed in Q1.3, Q1.12 and Q1.17. Q1.12, for example, uses
LS-NE identification of $M(q)$, $C(q)$ and $G(q)$, then runs a
\textbf{High-Order Finite-Time Observer (HOFFTO)} in joint space, using an FT
sensor only as ground truth. This yields \textbf{good joint-torque prediction
    and acceptable EE force estimates}, but still depends on accurate offline
dynamics. Finally, Q1.14 shows that combining LS-NE predicted torques with
measured joint torques enables \textbf{robust collision detection and
    localisation of the collided joint} using simple residual thresholds, again
assuming a reasonably accurate NRB model.

In summary, \textbf{Q1 methods show that classical LS-type identification and
    observers are mature and effective}:
\begin{itemize}
    \item \textbf{Mass and CoM} can be identified very reliably, even \textbf{without NRB and without FT sensors}.
    \item \textbf{Inertia} is consistently harder and requires \textbf{carefully designed dynamic excitation}, and still tends to be less accurate or weakly validated.
    \item \textbf{Torque prediction, contact detection and simple EE force estimation} are already at a high level with these methods, but they \textbf{rely on good friction modelling and reasonably accurate dynamics}.
\end{itemize}
