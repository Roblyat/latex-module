% \chapter{Abstract}
    {
    In the context of the digital factory, at UAS Technikum Vienna, where humans and robots share the tasks and the workspace, 
    the safe and efficient handling of payloads is essential. At the UAS digital factory payload is still handled without recognising 
    anything about the payloads internal parameters, leading to potential manipulation failures causing human harm. This study describes 
    the development of an advanced force/torque estimation method to improve a UR5 robots ability to recognize and handle different payload conditions. 
    This capability ensures the perception of the UR5 robot mounted on a mobile industrial robot platform to facilitate the safe and efficient transfer
    of payloads between different workspaces within the factory. The state of the art methods of force/torque estimation for industrial robots serve 
    neuronal networks and gaussian processes as the leading methods for accurate payload estimations. A gaussian process model has been developed to 
    estimate the forces and torques generated by the robot when executing trajectories. In a future project face, an awareness of payloads can be added 
    on the UR5 robot. In this way, the study aims to improve the intelligence of robotic systems in industrial environments and pave the way for higher 
    productivity and safety in digital manufacturing environments. This project face also yeelted in a simulation that provides a basis to record the 
    sensor data from the UR5's internal sensors and a force/torque sensor and a pipeline to train and evaluate  gaussian process models.
    }