% \chapter{Abstract}
	    {
	    In collaborative robotics, online payload estimation and safe interaction require accurate inverse-dynamics models under limited sensing.
	    This thesis investigates physics-informed learning of joint actuation from proprioceptive signals when only motor-current measurements are available.
	    A two-stage pipeline is proposed: a Deep Lagrangian Network (DeLaN) learns a mechanics-consistent baseline motor-current model from joint positions, velocities, and accelerations, and an LSTM sequence model is trained on the residual to capture history-dependent effects.
	    A trajectory-level preprocessing and split protocol, together with a best-model selection scheme that balances validation accuracy and seed stability, is introduced.
	    Experiments on the IEEE DataPort UR3e/UR10e datasets quantify how trajectory coverage affects convergence and generalisation and benchmark the approach against the published linear identification baseline.
	    The results show that increasing the number of trajectories reduces variability across dataset seeds and that the residual LSTM systematically decreases the remaining motor-current error of the DeLaN baseline.
	    On benchmark-scale splits ($50{,}000$ training and $5{,}000$ unseen test samples), the proposed DeLaN+LSTM model substantially outperforms the baseline under load and improves performance on the larger robot, while the baseline remains strongest on the unloaded UR3e condition.
	    The work provides an end-to-end, containerised implementation and reporting protocol for reproducible physics-structured inverse-dynamics learning in the motor-current domain.
	    }
