        \subsection{Limitations of the Current State of the Art}
        From the Q1 literature review in Section~\ref{sec:Q1_SoA_summary}, several limitations are particularly relevant for this work:
        \begin{itemize}
        \item \textbf{Separate treatment of robot, tool and payload.}
        Most approaches either perform robot dynamic parameter identification (RDPI) in joint or motor space~\cite{Q1_6_hu2025_fdrdi,Q1_13_dynamic_model_id_swevers2007,Q1_15_tang2023wls_rwspo},
        payload dynamic parameter identification (PDPI) in the FT frame~\cite{Q1_1_fast_inertial_id_cobots,Q1_5_payload_estimation_compensation_2025,Q1_7_10947736,Q1_10_duan_payload_ftsensor},
        or sensorless interaction-force estimation via observers~\cite{Q1_3_external_torque_smo,Q1_4_contact_force_kf,Q1_12_han_finite_time_observer,Q1_14_long2022sliding_momentum_observer,Q1_17_liu2021sensorless_dob_nn},
        but they rarely provide a unified view of robot, tool and payload dynamics.

        \item \textbf{Payload inertia is hard to identify robustly.}
        While payload mass (and often CoM) can be estimated accurately, inertia tensors are frequently weakly excited, poorly conditioned or only partially validated,
        in particular under cobot-safe excitation~\cite{Q1_1_fast_inertial_id_cobots,Q1_5_payload_estimation_compensation_2025,Q1_7_10947736,Q1_10_duan_payload_ftsensor,Q1_16_xu_payload_difference_2024}.

        \item \textbf{Heavy reliance on dedicated excitation and offline calibration.}
        Strong RDPI/PDPI results typically require carefully designed, long trajectories executed with and without payload and substantial offline processing~\cite{Q1_1_fast_inertial_id_cobots,Q1_5_payload_estimation_compensation_2025,Q1_6_hu2025_fdrdi,Q1_7_10947736,Q1_8_10944553,Q1_9_xu2022_double_weighting_payload_id,Q1_13_dynamic_model_id_swevers2007,Q1_15_tang2023wls_rwspo,Q1_16_xu_payload_difference_2024},
        which is at odds with continuous online awareness during everyday collaborative tasks.

        \item \textbf{Sensitivity to friction and transmission nonlinearities.}
        Observer-based force estimators and classical LS/NE pipelines depend on reasonably accurate friction and transmission models; residual errors increase around velocity reversals and at higher speeds,
        even when advanced observers or learned friction models are used~\cite{Q1_3_external_torque_smo,Q1_4_contact_force_kf,Q1_6_hu2025_fdrdi,Q1_12_han_finite_time_observer,Q1_15_tang2023wls_rwspo,Q1_17_liu2021sensorless_dob_nn}.
        \end{itemize}

        These limitations show that, within Q1, there is still no compact, online representation of robot, tool and payload dynamics that is robust to friction and transmission effects and applicable under normal cobot operating conditions.

        From the Q1 (classical / observer-based) literature, several structural limitations emerged:
        \begin{itemize}
        \item strong dependence on accurately identified base RBD models and friction compensation;
        \item payload inertia is systematically the weakest and least robustly validated quantity;
        \item most approaches rely on long, carefully designed excitation trajectories and offline identification;
        \item RDPI, PDPI and interaction-force estimation are typically treated in isolation rather than within a unified framework;
        \item robot, tool and payload dynamics are rarely represented as a single, continuously updated notion of ``self-awareness'' in the measurement frame.
        \end{itemize}

        The Q2 (Gaussian Process–based) works add further limitations:
        \begin{itemize}
        \item GPs are used as lumped residual models on top of a nominal RBD, so robot, tool, payload, friction and contact effects are entangled in a single disturbance term rather than being identified explicitly;
        \item all GPs are trained offline on non-contact data and then deployed with only limited adaptation, so robustness to changing tools, payloads and task distributions is largely unexplored;
        \item the focus is on torque prediction and binary contact detection; none of the methods provide a structured, physics-consistent estimate of the end-effector wrench or of payload/tool inertial parameters that could serve as a basis for precise tool/gripper compensation and subsequent PDPI.
        \end{itemize}