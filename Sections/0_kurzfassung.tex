% \chapter{Kurzfassung}
	    {
	    In der kollaborativen Robotik erfordern Online-Nutzlasterkennung und sichere Interaktion genaue Inversdynamikmodelle bei eingeschränkter Sensorik.
	    Diese Arbeit untersucht das physikinspirierte Lernen der Gelenkaktuation aus propriozeptiven Signalen, wenn nur Motorstrommessungen verfügbar sind.
	    Es wird eine zweistufige Pipeline vorgeschlagen: Ein Deep Lagrangian Network (DeLaN) lernt ein mechanikkonsistentes Basismodell für die Motorströme aus Gelenkpositionen, -geschwindigkeiten und -beschleunigungen, und ein LSTM-Sequenzmodell wird auf dem Residuum trainiert, um geschichtsabhängige Effekte abzubilden.
	    Dazu werden ein Trajektorien-basiertes Vorverarbeitungs- und Split-Protokoll sowie ein Best-Model-Selektionsschema eingeführt, das Validierungsgenauigkeit und Seed-Stabilität ausbalanciert.
	    Experimente auf den IEEE-DataPort-Datensätzen UR3e/UR10e quantifizieren, wie die Trajektorienabdeckung Konvergenz und Generalisierung beeinflusst, und benchmarken den Ansatz gegen die publizierte lineare Identifikations-Baseline.
	    Die Ergebnisse zeigen, dass eine höhere Anzahl von Trajektorien die Variabilität über Datensatz-Seeds reduziert und dass das Residual-LSTM den verbleibenden Motorstromfehler des DeLaN-Baselinesystems systematisch verringert.
	    Bei benchmark-großen Splits ($50{,}000$ Trainings- und $5{,}000$ ungesehene Test-Samples) übertrifft das vorgeschlagene DeLaN+LSTM-Modell die Baseline unter Last deutlich und verbessert die Performance auf dem größeren Roboter, während die Baseline im unbelasteten UR3e-Fall weiterhin am stärksten ist.
	    Die Arbeit stellt eine durchgängige, containerisierte Implementierung und ein Reporting-Protokoll für reproduzierbares, physikstrukturiertes Inversdynamik-Lernen im Motorstrombereich bereit.
	    }
