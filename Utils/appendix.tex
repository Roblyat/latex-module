%%%%%%%%%%%%%%%%%%%%%%%%%%%%%%%%%%%%%%%% Appendix
\clearpage
\appendix
\chapter{Appendix Kinematic and Dynamic Background of Robot Manipulation and Environment Interaction}
The external wrench $\vec{F}_{\mathrm{ext}}$ in~\eqref{eq:tau_ext} is a
6-dimensional vector expressed in the sensor/tool frame $S$,
\begin{equation}
  \vec{F}_{\mathrm{ext}}
  =
  \begin{bmatrix}
    \mathbf{f} \\[2pt] \boldsymbol{\tau}
  \end{bmatrix}
  \in \mathbb{R}^6,
\end{equation}
with $\mathbf{f} \in \mathbb{R}^3$ the linear force and
$\boldsymbol{\tau} \in \mathbb{R}^3$ the moment about the frame origin.
For a rigid body with parameters $\boldsymbol{\phi}_{\mathrm{eff}}$
(mass, CoM and inertia) moving with linear and angular motion
$(\mathbf{a}, \boldsymbol{\alpha}, \boldsymbol{\omega})$, the Newton--Euler
equations (cf.~\eqref{eq:newtonEuler}) give
\begin{equation}
  \begin{bmatrix}
    \mathbf{f} \\[2pt] \boldsymbol{\tau}
  \end{bmatrix}
  =
  m
  \begin{bmatrix}
    \mathbf{I} & -[\mathbf{c}]^{\times} \\
    [\mathbf{c}]^{\times} & \mathbf{J}_s
  \end{bmatrix}
  \begin{bmatrix}
    \mathbf{a} \\[2pt] \boldsymbol{\alpha}
  \end{bmatrix}
  +
  \begin{bmatrix}
    m[\boldsymbol{\omega}]^{\times}[\boldsymbol{\omega}]^{\times}\mathbf{c} \\
    [\boldsymbol{\omega}]^{\times}\mathbf{J}_s\boldsymbol{\omega}
  \end{bmatrix}.
\end{equation}

The translational part $\mathbf{f}$ can be written as
\begin{equation}
  \mathbf{f}
  =
  m\mathbf{a}
  - m[\mathbf{c}]^{\times}\boldsymbol{\alpha}
  + m[\boldsymbol{\omega}]^{\times}[\boldsymbol{\omega}]^{\times}\mathbf{c},
  \label{eq:f_expanded}
\end{equation}
where the first term $m\mathbf{a}$ is the familiar inertial force, while
$- m[\mathbf{c}]^{\times}\boldsymbol{\alpha}$ and
$m[\boldsymbol{\omega}]^{\times}[\boldsymbol{\omega}]^{\times}\mathbf{c}$
collect the additional centripetal and Coriolis contributions induced by the angular motion
$\boldsymbol{\omega}$ and the CoM offset $\mathbf{c}$. 
Similarly, the rotational part $\boldsymbol{\tau}$ can be written as
\begin{equation}
  \boldsymbol{\tau}
  =
  m[\mathbf{c}]^{\times}\mathbf{a}
  +
  \mathbf{J}_s\boldsymbol{\alpha}
  +
  [\boldsymbol{\omega}]^{\times}\mathbf{J}_s\boldsymbol{\omega},
  \label{eq:tau_expanded}
\end{equation}
where $\mathbf{J}_s\boldsymbol{\alpha}$ is the inertial moment due to angular acceleration,
$m[\mathbf{c}]^{\times}\mathbf{a}$ is the torque induced by the translational acceleration of the offset CoM,
and $[\boldsymbol{\omega}]^{\times}\mathbf{J}_s\boldsymbol{\omega}$ represents gyroscopic effects associated with the angular velocity $\boldsymbol{\omega}$.

In compact form, for a
given motion $(\mathbf{a},\boldsymbol{\alpha},\boldsymbol{\omega})$ this can
be written as
\begin{equation}
  \vec{F}_{\mathrm{ext}}
  =
  \vec{F}_{\mathrm{dyn}}(\mathbf{a},\boldsymbol{\alpha},\boldsymbol{\omega};
                         \boldsymbol{\phi}_{\mathrm{eff}})
  =
  Y(\mathbf{a},\boldsymbol{\alpha},\boldsymbol{\omega})\,
  \boldsymbol{\phi}_{\mathrm{eff}},
\end{equation}
where $Y(\cdot)$ is a $6\times 10$ regressor matrix that is linear in
$\boldsymbol{\phi}_{\mathrm{eff}}$ but nonlinear in the motion variables.
Hence, $\vec{F}_{\mathrm{ext}}$ is not simply $m\mathbf{a}$, nor can it be
written as $\boldsymbol{\phi}_{\mathrm{eff}}\ddot{\mathbf{q}}$; the mapping
from joint accelerations $\ddot{\mathbf{q}}$ to $\vec{F}_{\mathrm{ext}}$
passes through the robot kinematics and the Newton--Euler relations.

Once the wrench at the flange is known, the corresponding joint torques are
obtained via
\begin{equation}
  \boldsymbol{\tau}_{\mathrm{ext}}
  =
  {}^{S}\!J(\mathbf{q})^{\top}\,\vec{F}_{\mathrm{ext}},
\end{equation}
where ${}^{S}\!J(\mathbf{q})$ is the Jacobian of the sensor/tool frame $S$.
Combining the relations above yields the identification-friendly form
\begin{equation}
  \boldsymbol{\tau}_{\mathrm{ext}}
  =
  \mathbf{J}^T(\mathbf{q})\,
  Y(\mathbf{a},\boldsymbol{\alpha},\boldsymbol{\omega})\,
  \boldsymbol{\phi}_{\mathrm{eff}},
\end{equation}
which makes explicit that $\boldsymbol{\tau}_{\mathrm{ext}}$ is linear in
$\boldsymbol{\phi}_{\mathrm{eff}}$, but nonlinear in
$\mathbf{q},\dot{\mathbf{q}},\ddot{\mathbf{q}}$ through the dependence on
$\mathbf{a},\boldsymbol{\alpha},\boldsymbol{\omega}$.




\clearpage
\chapter{Appendix B}