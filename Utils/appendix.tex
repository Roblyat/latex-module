%%%%%%%%%%%%%%%%%%%%%%%%%%%%%%%%%%%%%%%% Appendix
\clearpage
\appendix
\chapter{Kinematic and Dynamic Background of Robot Manipulation and Environment Interaction}
The external wrench $\vec{F}_{\mathrm{ext}}$ in~\eqref{eq:tau_ext} is a
6-dimensional vector expressed in the sensor/tool frame $S$,
\begin{equation}
  \vec{F}_{\mathrm{ext}}
  =
  \begin{bmatrix}
    \mathbf{f} \\[2pt] \boldsymbol{\tau}
  \end{bmatrix}
  \in \mathbb{R}^6,
\end{equation}
with $\mathbf{f} \in \mathbb{R}^3$ the linear force and
$\boldsymbol{\tau} \in \mathbb{R}^3$ the moment about the frame origin.
For a rigid body with parameters $\boldsymbol{\phi}_{\mathrm{eff}}$
(mass, CoM and inertia) moving with linear and angular motion
$(\mathbf{a}, \boldsymbol{\alpha}, \boldsymbol{\omega})$, the Newton--Euler
equations (cf.~\eqref{eq:newtonEuler}) give
\begin{equation}
  \begin{bmatrix}
    \mathbf{f} \\[2pt] \boldsymbol{\tau}
  \end{bmatrix}
  =
  m
  \begin{bmatrix}
    \mathbf{I} & -[\mathbf{c}]^{\times} \\
    [\mathbf{c}]^{\times} & \mathbf{J}_s
  \end{bmatrix}
  \begin{bmatrix}
    \mathbf{a} \\[2pt] \boldsymbol{\alpha}
  \end{bmatrix}
  +
  \begin{bmatrix}
    m[\boldsymbol{\omega}]^{\times}[\boldsymbol{\omega}]^{\times}\mathbf{c} \\
    [\boldsymbol{\omega}]^{\times}\mathbf{J}_s\boldsymbol{\omega}
  \end{bmatrix}.
\end{equation}

The translational part $\mathbf{f}$ can be written as
\begin{equation}
  \mathbf{f}
  =
  m\mathbf{a}
  - m[\mathbf{c}]^{\times}\boldsymbol{\alpha}
  + m[\boldsymbol{\omega}]^{\times}[\boldsymbol{\omega}]^{\times}\mathbf{c},
  \label{eq:f_expanded}
\end{equation}
where the first term $m\mathbf{a}$ is the familiar inertial force, while
$- m[\mathbf{c}]^{\times}\boldsymbol{\alpha}$ and
$m[\boldsymbol{\omega}]^{\times}[\boldsymbol{\omega}]^{\times}\mathbf{c}$
collect the additional centripetal and Coriolis contributions induced by the angular motion
$\boldsymbol{\omega}$ and the CoM offset $\mathbf{c}$. 
Similarly, the rotational part $\boldsymbol{\tau}$ can be written as
\begin{equation}
  \boldsymbol{\tau}
  =
  m[\mathbf{c}]^{\times}\mathbf{a}
  +
  \mathbf{J}_s\boldsymbol{\alpha}
  +
  [\boldsymbol{\omega}]^{\times}\mathbf{J}_s\boldsymbol{\omega},
  \label{eq:tau_expanded}
\end{equation}
where $\mathbf{J}_s\boldsymbol{\alpha}$ is the inertial moment due to angular acceleration,
$m[\mathbf{c}]^{\times}\mathbf{a}$ is the torque induced by the translational acceleration of the offset CoM,
and $[\boldsymbol{\omega}]^{\times}\mathbf{J}_s\boldsymbol{\omega}$ represents gyroscopic effects associated with the angular velocity $\boldsymbol{\omega}$.

In compact form, for a
given motion $(\mathbf{a},\boldsymbol{\alpha},\boldsymbol{\omega})$ this can
be written as
\begin{equation}
  \vec{F}_{\mathrm{ext}}
  =
  \vec{F}_{\mathrm{dyn}}(\mathbf{a},\boldsymbol{\alpha},\boldsymbol{\omega};
                         \boldsymbol{\phi}_{\mathrm{eff}})
  =
  Y(\mathbf{a},\boldsymbol{\alpha},\boldsymbol{\omega})\,
  \boldsymbol{\phi}_{\mathrm{eff}},
\end{equation}
where $Y(\cdot)$ is a $6\times 10$ regressor matrix that is linear in
$\boldsymbol{\phi}_{\mathrm{eff}}$ but nonlinear in the motion variables.
Hence, $\vec{F}_{\mathrm{ext}}$ is not simply $m\mathbf{a}$, nor can it be
written as $\boldsymbol{\phi}_{\mathrm{eff}}\ddot{\mathbf{q}}$; the mapping
from joint accelerations $\ddot{\mathbf{q}}$ to $\vec{F}_{\mathrm{ext}}$
passes through the robot kinematics and the Newton--Euler relations.

Once the wrench at the flange is known, the corresponding joint torques are
obtained via
\begin{equation}
  \boldsymbol{\tau}_{\mathrm{ext}}
  =
  {}^{S}\!J(\mathbf{q})^{\top}\,\vec{F}_{\mathrm{ext}},
\end{equation}
where ${}^{S}\!J(\mathbf{q})$ is the Jacobian of the sensor/tool frame $S$.
Combining the relations above yields the identification-friendly form
\begin{equation}
  \boldsymbol{\tau}_{\mathrm{ext}}
  =
  \mathbf{J}^T(\mathbf{q})\,
  Y(\mathbf{a},\boldsymbol{\alpha},\boldsymbol{\omega})\,
  \boldsymbol{\phi}_{\mathrm{eff}},
\end{equation}
which makes explicit that $\boldsymbol{\tau}_{\mathrm{ext}}$ is linear in
$\boldsymbol{\phi}_{\mathrm{eff}}$, but nonlinear in
$\mathbf{q},\dot{\mathbf{q}},\ddot{\mathbf{q}}$ through the dependence on
$\mathbf{a},\boldsymbol{\alpha},\boldsymbol{\omega}$.


\clearpage
\chapter{Query Categories - Index Terms}
\label{app:query_categories}

This appendix lists the index terms used to construct the query categories
illustrated in Fig.~\ref{fig:query_logic}. The original nine term groups were
consolidated into five content clusters $C_1,\dots,C_5$ and the goal/context
term sets $C_{mt}$ and $C_{ct}$.

\section*{Content Clusters $C_i$}

\subsubsection*{$C_1$: Classical / Observers}

\begin{itemize}
  \item momentum observer (MO)
  \item generalized momentum observer (GMO)
  \item disturbance observer (DOB)
  \item reaction force observer (RFOB)
  \item Kalman filter (KF)
  \item extended Kalman filter (EKF)
  \item unscented Kalman filter (UKF)
  \item state observer
  \item least squares (LS)
  \item weighted least squares (WLS)
  \item iterative reweighted least squares (IRLS)
  \item recursive least squares (RLS)
  \item momentum-based observer
  \item dynamic state observer
  \item observer
  \item force observer
  \item torque observer
\end{itemize}

\subsubsection*{$C_2$: Gaussian Process (GP)}

\begin{itemize}
  \item gaussian process regression (GPR)
  \item sparse gaussian process (SGP, SGPR)
  \item multi-output gaussian process (MOGP)
  \item multi-task gaussian process (MTGP)
  \item gaussian process state space model (GPSSM)
  \item hybrid gaussian process
  \item GP residual
  \item gaussian process dynamics
  \item GP inverse dynamics
  \item bayesian nonparametric regression (BNPR)
\end{itemize}

\subsubsection*{$C_3$: Deep Sequence Models (MLP / GRU / TCN / Transformer / LSTM)}

\begin{itemize}
  % original deep sequence model terms
  \item neural network inverse dynamics (NN-ID)
  \item deep learning
  \item multi layer perceptron (MLP)
  \item residual network (ResNet)
  \item long short-term memory (LSTM)
  \item gated recurrent unit (GRU)
  \item temporal convolutional network (TCN)
  \item causal convolution
  \item dilated convolution
  \item transformer model
  \item attention model
  \item sequence-to-sequence (seq2seq, S2S)
  \item sequence GAN (SeqGAN, TimeGAN)
  \item GAN
  \item Generative Adversarial Networks
  % from Hybrid/Residual: network-architecture side
  \item residual neural network (ResNN)
  \item residual GAN
  % from Domain Adaptation / Latent Context
  \item domain adaptation (DA)
  \item transfer learning (TL)
  \item meta learning (ML)
  \item context variable dynamics
  \item latent variable model (LVM)
  \item amortized inference (AI)
  \item test time adaptation (TTA)
  \item online adaptation (OA)
  \item feature invariance
  \item domain invariant features (DIF)
  \item few shot learning (FSL)
  \item zero shot transfer (ZSL)
  % from Reinforcement Learning
  \item reinforcement
  \item reinforcement learning
  \item Isaac Gym differentiable
  \item Isaac Lab differentiable
  \item Isaac Gym
  \item Isaac Lab
\end{itemize}

\subsubsection*{$C_4$: Physics-Informed / Differentiable}

\begin{itemize}
  % from Hybrid / Residual (dynamics-structure side)
  \item residual learning dynamics
  \item hybrid model dynamics
  \item analytical dynamics neural network (ADNN)
  \item physics residual
  \item rigid body dynamics residual (RBD residual)
  \item Newton Euler residual (NE residual)
  \item nominal dynamics model (NDM)
  \item neural correction
  \item learning inverse dynamics residual (ID residual)
  % original Physics-Informed / Differentiable terms
  \item physics-informed neural network (PINN)
  \item differentiable physics
  \item differentiable simulation (DiffSim)
  \item differentiable robot model
  \item differentiable dynamics
  \item neural ODE (NODE)
  \item torchdiffeq
  \item ODE-net
  \item physics-guided machine learning robotics (PGML)
\end{itemize}

\subsubsection*{$C_5$: Surveys}

\begin{itemize}
  \item survey
  \item benchmarking
  \item review
  \item overview
  \item systematic comparison
\end{itemize}

\section*{Goal \& Domain Terms $C_T$}

\subsubsection*{$C_{mt}$: Estimation \& Modeling Terms}

\begin{itemize}
  \item external force
  \item force measurement
  \item force estimation
  \item force/torque estimation
  \item wrench estimation
  \item joint torque estimation
  \item end-effector force
  \item end-effector torque
  \item inertial parameters
  \item inertial parameter identification (IPI)
  \item online payload identification
  \item payload identification
  \item payload estimation
  \item object parameter estimation
  \item parameter identification
  \item inertia tensor
  \item inertia tensor estimation
  \item center of mass (CoM)
  \item rigid body dynamics
  \item friction approximation
  \item nonlinear friction model
  \item external perturbations
  \item force torque sensor (F/T sensor)
  \item external force estimation (EFE)
  \item external torque estimation (ETE)
  \item torque estimation
  \item parameter identification differentiable simulation
  \item payload identification (PI)
  \item payload estimation (PE)
  \item contact force
  % broader, more generic terms used only in combination
  \item nonlinear systems
  \item noise
  \item signal noise
  \item noise estimation
\end{itemize}

Note that the last four entries (\emph{nonlinear systems}, \emph{noise}, \emph{signal noise}, \emph{noise estimation}) are generic terms that occur across many physical systems beyond robotic manipulators.
Including them in the queries significantly increased the number of retrieved results.

\subsubsection*{$C_{ct}$: Robotics Context Terms}

\begin{itemize}
  \item robotic manipulator
  \item robotic arm
  \item robotic manipulation
  \item robot payload
\end{itemize}

%%%%%%%%%%%%%%%%%%%%%%%%%%%%%%%%%%%%%%%%%%%%%%%%
\chapter{Concept Graph}
\label{soa_references_graph}

\begin{table}[H]
\centering
\caption{Overview of selected references by category, source and citation count. (accessed 2025-11-30T18:53:00 [YYYY-MM-DDTHH:mm:ss])}
\label{tab:soa_references}
\begin{tabular}{llllr}
\toprule
\textbf{Reference} & \textbf{Cite} & \textbf{Year} & \textbf{Database} & \textbf{Citations} \\
\midrule
R1  & \cite{Q1_1_fast_inertial_id_cobots}                & 2022 & IEEE          & 13  \\
R2  & \cite{Q1_2_online_payload_mo}                      & 2022 & IEEE          & 16  \\
R3  & \cite{Q1_3_external_torque_smo}                    & 2022 & IEEE          & 47  \\
R4  & \cite{Q1_4_contact_force_kf}                       & 2021 & ScienceDirect & 34  \\
R5  & \cite{Q1_5_payload_estimation_compensation_2025}   & 2025 & IEEE          & 0   \\
R6  & \cite{Q1_6_hu2025_fdrdi}                           & 2025 & IEEE          & 2   \\
R7  & \cite{Q1_7_10947736}                               & 2025 & IEEE          & 1   \\
R8  & \cite{Q1_8_10944553}                               & 2025 & IEEE          & 0   \\
R9  & \cite{Q1_9_xu2022_double_weighting_payload_id}     & 2022 & Cambridge     & 10  \\
R10 & \cite{Q1_10_duan_payload_ftsensor}                 & 2022 & MPDI          & 27  \\
R11 & \cite{Q1_11_wei_composite_filter}                  & 2025 & IEEE          & 2   \\
R12 & \cite{Q1_12_han_finite_time_observer}              & 2022 & IEEE          & 57  \\
R13 & \cite{Q1_13_dynamic_model_id_swevers2007}          & 2007 & IEEE          & 303 \\
R14 & \cite{Q1_14_long2022sliding_momentum_observer}     & 2022 & MPDI          & 17  \\
R15 & \cite{Q1_15_tang2023wls_rwspo}                     & 2023 & MPDI          & 16  \\
R16 & \cite{Q1_16_xu_payload_difference_2024}            & 2024 & Cambridge     & 2   \\
R17 & \cite{Q1_17_liu2021sensorless_dob_nn}              & 2021 & ScienceDirect & 100 \\
\midrule
R18 & \cite{Q2_1_contact_force_gp_observer}              & 2022 & IEEE          & 3   \\
R19 & \cite{Q2_2_contact_force_gpadkf}                   & 2024 & IEEE          & 19  \\
R20 & \cite{Q2_3_contact_detection_gp}                   & 2022 & IEEE          & 2   \\
R21 & \cite{Q2_4_GIACOMUZZO20231584}                     & 2023 & ScienceDirect & 2   \\
\midrule
R22 & \cite{Q3_1_tao_bll}                                & 2025 & Cambridge     & 1   \\
R23 & \cite{Q3_2_encoder_attention_payload}              & 2023 & IEEE          & 3   \\
R24 & \cite{Q3_3_lstm_force_estimation}                  & 2021 & MPDI          & 10  \\
R25 & \cite{Q3_4_asgrnn_force_observer}                  & 2023 & ScienceDirect & 9   \\
R26 & \cite{Q3_5_contact_localization_cnn}               & 2021 & IEEE          & 5   \\
R27 & \cite{Q3_6_payload_id_incremental_ensemble}        & 2024 & IEEE          & 1   \\
R28 & \cite{Q3_7_payload_id_online_ensemble}             & 2024 & IEEE          & 13  \\
R29 & \cite{Q3_8_payload_id_catastrophic_forgetting}     & 2024 & Frontiersin   & 0   \\
\midrule
R30 & \cite{Q4_1_extended_delan_motor}                   & 2025 & IEEE          & 0   \\
R31 & \cite{Q4_2_lutter2023combiningphysicsdeeplearning} & 2023 & ArXiv         & -   \\
R32 & \cite{Q4_3_residual_pinns_dynamics_id}             & 2025 & IEEE          & 0   \\
R33 & \cite{Q4_4_10729277}                               & 2024 & IEEE          & 10  \\
R34 & \cite{Q4_5_10305255}                               & 2023 & IEEE          & 27  \\
\bottomrule
\end{tabular}
\end{table}

%%%%%%%%%%%%%%%%%%%%%%%%%%%%%%%%%%%%%%%%%%%%%%%%%%%
\chapter{DeLaN parameterisation and friction model}
\label{ch:appendix_delan}

This appendix details the specific neural parameterisation of the Deep Lagrangian
Network (DeLaN) used in Stage~1 of the proposed architecture, i.e.\ the
construction of the inertia, potential/gravity and friction subnetworks that
together implement the inverse-dynamics mapping
$f_{\mathrm{DeLaN}}(\mathbf{q},\dot{\mathbf{q}},\ddot{\mathbf{q}};\boldsymbol{\theta},\boldsymbol{\psi})$
described in Section~\ref{sec:methods}.

\section{Inertia subnetwork}
\label{sec:app_inertia}

Following the improved DeLaN formulation of~\cite{Q4_6_HU2026103093} and its PINN-based extension to industrial robots in~\cite{Q4_4_10729277},
the symmetric positive-definite inertia matrix
$\mathbf{M}_{\boldsymbol{\theta}}(\mathbf{q}) \in \mathbb{R}^{n\times n}$ is
obtained from a learned Cholesky factor.
We parameterise a lower–triangular matrix
$\hat{\mathbf{L}}(\mathbf{q};\boldsymbol{\theta}_M)$ and set
\begin{equation}
  \mathbf{M}_{\boldsymbol{\theta}}(\mathbf{q})
  =
  \hat{\mathbf{L}}(\mathbf{q};\boldsymbol{\theta}_M)\,
  \hat{\mathbf{L}}(\mathbf{q};\boldsymbol{\theta}_M)^{\top}.
  \label{eq:app_M_from_L}
\end{equation}
The factor $\hat{\mathbf{L}}$ is decomposed into a strictly lower–triangular
part and a diagonal part,
\begin{equation}
  \hat{\mathbf{L}}(\mathbf{q};\boldsymbol{\theta}_M)
  =
  \mathbf{L}_o(\mathbf{q};\boldsymbol{\theta}_o)
  +
  \mathbf{L}_d(\mathbf{q};\boldsymbol{\theta}_d),
\end{equation}
where both $\mathbf{L}_o$ and $\mathbf{L}_d$ are represented by multilayer
perceptrons (MLPs) taking $\mathbf{q}$ as input:

\begin{itemize}
  \item $\mathbf{L}_o(\mathbf{q};\boldsymbol{\theta}_o)$ is lower–triangular
        with zero diagonal entries and uses a linear output layer.
  \item $\mathbf{L}_d(\mathbf{q};\boldsymbol{\theta}_d)$ is diagonal.
        Its diagonal elements are obtained as
        \[
          [\mathbf{L}_d]_{ii}
          =
          \mathrm{ReLU}\bigl(h_i(\mathbf{q})\bigr) + \varepsilon,
        \]
        where $h_i$ is the $i$-th output of an MLP and $\varepsilon>0$ is a
        small constant. The ReLU+offset guarantees strictly positive diagonal
        entries and hence $\mathbf{M}_{\boldsymbol{\theta}}(\mathbf{q}) \succ 0$
        for all $\mathbf{q}$.
\end{itemize}

In practice, all MLPs in the DeLaN core (the two inertia subnetworks and the
potential network below) use sinusoidal activation functions as suggested
in~\cite{Q4_4_10729277,Q4_6_HU2026103093}, since for revolute manipulators the elements of
$\mathbf{M}(\mathbf{q})$ can be written as linear combinations of
$\sin(\cdot)$ and $\cos(\cdot)$.

\section{Potential and gravity subnetworks}
\label{sec:app_potential}

The conservative part of the dynamics is encoded by a learned Lagrangian
\begin{equation}
  \mathcal{L}_{\boldsymbol{\theta}}(\mathbf{q},\dot{\mathbf{q}})
  =
  \tfrac{1}{2}\,\dot{\mathbf{q}}^{\top}
  \mathbf{M}_{\boldsymbol{\theta}}(\mathbf{q})\,\dot{\mathbf{q}}
  -
  V_{\boldsymbol{\theta}_V}(\mathbf{q}),
\end{equation}
where $V_{\boldsymbol{\theta}_V}(\mathbf{q})$ is represented by another MLP
(the “potential” subnetwork) with parameters $\boldsymbol{\theta}_V$.
Using automatic differentiation we obtain the gravity term as the gradient of
the potential,
\begin{equation}
  \mathbf{G}_{\boldsymbol{\theta}}(\mathbf{q})
  =
  \frac{\partial V_{\boldsymbol{\theta}_V}(\mathbf{q})}{\partial \mathbf{q}}
  \approx
  \mathbf{g}(\mathbf{q}).
\end{equation}
This DeLaN-style representation of the potential and its gradient follows the
improved formulations proposed in~\cite{Q4_6_HU2026103093,Q4_4_10729277}, where
the potential is modelled by a neural network and the gravity vector is
obtained as its configuration-space gradient.

Together with the inertia matrix~\eqref{eq:app_M_from_L}, the conservative
torques are given by
\begin{equation}
  \boldsymbol{\tau}_{\mathrm{cons}}
  (\mathbf{q},\dot{\mathbf{q}},\ddot{\mathbf{q}};\boldsymbol{\theta})
  =
  \mathbf{M}_{\boldsymbol{\theta}}(\mathbf{q})\,\ddot{\mathbf{q}}
  +
  \mathbf{C}_{\boldsymbol{\theta}}(\mathbf{q},\dot{\mathbf{q}})\,\dot{\mathbf{q}}
  +
  \mathbf{G}_{\boldsymbol{\theta}}(\mathbf{q}),
\end{equation}
with the Coriolis/centrifugal term
$\mathbf{C}_{\boldsymbol{\theta}}(\mathbf{q},\dot{\mathbf{q}})\dot{\mathbf{q}}$
obtained from $\mathcal{L}_{\boldsymbol{\theta}}$ via the Euler–Lagrange
equations.

\section{Friction subnetwork}
\label{sec:app_friction}

Joint friction and other non-conservative effects are modelled in a physically
interpretable way by a Coulomb–viscous law, following the improved DeLaN
formulation in~\cite{Q4_6_HU2026103093} and empirical studies on UR robots
with Coulomb–viscous friction~\cite{QM_2_Coulomb_viscous_friction_ur5_10610737}.
For each joint $i$ we use
\begin{equation}
  \tau_{\mathrm{fric},i}(\dot q_i;\boldsymbol{\psi})
  =
  f_{c,i}\,\mathrm{sgn}(\dot q_i)
  +
  f_{v,i}\,\dot q_i,
\end{equation}
where $f_{c,i}$ and $f_{v,i}$ are the Coulomb and viscous friction
coefficients for joint $i$. Collecting these in vectors
$\mathbf{f}_c,\mathbf{f}_v \in \mathbb{R}^n$ yields the compact vector form
\begin{equation}
  \boldsymbol{\tau}_{\mathrm{fric}}
  (\dot{\mathbf{q}};\boldsymbol{\psi})
  =
  \mathbf{f}_c \,\mathrm{sgn}(\dot{\mathbf{q}})
  +
  \mathbf{f}_v \,\dot{\mathbf{q}},
  \label{eq:app_coulomb_viscous}
\end{equation}
with $\mathrm{sgn}(\cdot)$ applied element-wise. This can be interpreted as
a joint-wise affine map
\[
  \boldsymbol{\tau}_{\mathrm{fric}}
  =
  f_{\mathrm{fric}}\bigl([\dot{\mathbf{q}},\mathrm{sgn}(\dot{\mathbf{q}})];\boldsymbol{\psi}\bigr),
\]
consistent with the main Methods section.

Equation~\eqref{eq:app_coulomb_viscous} corresponds directly to the friction
term in~\cite{Q4_6_HU2026103093}, where
$\tau_f = f_c \,\mathrm{sgn}(\dot{q}) + f_v \dot{q}$ is added to the DeLaN
prediction.

\section{Resulting DeLaN inverse-dynamics map}
\label{sec:app_delan_map}

Combining the inertia, potential/gravity and friction subnetworks, the DeLaN
inverse-dynamics model used in this thesis is
\begin{equation}
  \hat{\boldsymbol{\tau}}_{\mathrm{DeLaN}}
  (\mathbf{q},\dot{\mathbf{q}},\ddot{\mathbf{q}};\boldsymbol{\theta},\boldsymbol{\psi})
  =
  \boldsymbol{\tau}_{\mathrm{cons}}
    (\mathbf{q},\dot{\mathbf{q}},\ddot{\mathbf{q}};\boldsymbol{\theta})
  +
  \boldsymbol{\tau}_{\mathrm{fric}}(\dot{\mathbf{q}};\boldsymbol{\psi})
  =
  f_{\mathrm{DeLaN}}(\mathbf{q},\dot{\mathbf{q}},\ddot{\mathbf{q}};\boldsymbol{\theta},\boldsymbol{\psi}),
\end{equation}
which is trained in Stage~1 by minimising the joint-space loss
\begin{equation}
  \mathcal{L}_{\mathrm{DeLaN}}(\boldsymbol{\theta},\boldsymbol{\psi})
  =
  \frac{1}{N}\sum_{k=1}^{N}
  \left\|
    f_{\mathrm{DeLaN}}
      (\mathbf{q}_k,\dot{\mathbf{q}}_k,\ddot{\mathbf{q}}_k;
       \boldsymbol{\theta},\boldsymbol{\psi})
    -
    \boldsymbol{\tau}_{\mathrm{motor},k}
  \right\|_2^2,
\end{equation}
with $\boldsymbol{\tau}_{\mathrm{motor},k} = k_t \boldsymbol{I}_k$ as defined
in~\eqref{eq:tau_motor}. The trained DeLaN then serves as the nominal
inverse-dynamics backbone that is subsequently refined by the residual LSTM
in Stage~2, as described in Section~\ref{sec:methods}.
