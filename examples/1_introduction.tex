\chapter{Introduction}
    
    Robots play a central role in modern industry and manufacturing, but with increasing complexity of tasks and the need for closer human-robot collaboration, new challenges arise. Safe manipulation of payloads and safe interaction with humans are the two main challenges in human-robot collaboration. Both require the knowledge of the inertial parameters of the payload. The ability to recognize different payload conditions advantages a safer human-robot collaboration. The primary objective of this analysis is to estimate the mass $\boldsymbol{m}$ and inertia tensor $\boldsymbol{J}$ of the manipulated payload, $\phi_{\text{payload}}$~\eqref{eq:rigidBody}. The manipulated payloads considered here are geometrically simple shapes—namely, a cube and a cylinder—where the center of mass is located at the geometric center of each rigid body. The contribution of this study is to advance human-robot collaboration safety by enabling robots to estimate the mass and inertia tensor of payloads in real-time. Awareness of these inertial parameters is crucial for mitigating risks and ensuring safe manipulation in close proximity to humans, a priority recognized by recent advancements in human-robot interaction research \cite{bai2024sensorless, Popov2019, nadeau2022fast, kurdas2022online, su2021deep, haninger2022model, HANINGER2023104431}.

    \vspace{1em}
    At first there are two notations that are important in this case.
    
    \begin{equation}
    \phi^T = \begin{bmatrix} m & m c_x & m c_y & m c_z \\
    J_{xx} & J_{xy} & J_{xz} & J_{yy} & J_{yz} & J_{zz} \end{bmatrix} \in \mathbb{R}^{10}
    \label{eq:rigidBody}
    \end{equation}
    
    The vector $\phi$ describes the internal parameter properties of a rigid body.
    
    \begin{equation}
    \begin{bmatrix} \mathbf{f} \\ \boldsymbol{\tau} \end{bmatrix} = m \begin{bmatrix} \mathbf{I}_{3 \times 3} & -[\mathbf{c}]^\times \\ [\mathbf{c}]^\times & \mathbf{J}_s \end{bmatrix} \begin{bmatrix} \mathbf{a} \\ \boldsymbol{\alpha} \end{bmatrix} + \begin{bmatrix} m[\boldsymbol{\omega}]^\times[\boldsymbol{\omega}]^\times\mathbf{c} \\ [\boldsymbol{\omega}]^\times \mathbf{J}_s \boldsymbol{\omega} \end{bmatrix}
    \label{eq:newtonEuler}
    \end{equation}
    
    The Newton-Euler equations describe how the forces and torques acting on a rigid body are related to its acceleration (\(\mathbf{a}\)), angular acceleration (\(\boldsymbol{\alpha}\))  and angular velocity (\(\boldsymbol{\omega}\)), taking into account its mass properties and moments of inertia.
    
    At this point a effective rigid body is introduced. It is described as \(\phi_{\text{effective}}\)~\eqref{eq:rigidBody} and is differentiated in \(\phi_{\text{payload}}\) and \(\phi_{\text{gripper}}\).
    
    \begin{equation}
    \phi_{\text{effective}} = \phi_{\text{payload}} + \phi_{\text{gripper}}
    \label{eq:rigidEffective}
    \end{equation}
    
    Since a force/torque sensor is mounted between the robot flange and the gripper, the Newton/Euler equations~\eqref{eq:newtonEuler} can be used to determine the internal parameters of \(\phi_{\text{effective}}\) \cite{kurdas2022online, nadeau2022fast}. We can measure the force/torque (f/t) data at the measurement frame (MF) of the f/t sensor and use more sensor data from motor sensors to get \(\mathbf{a}\), \(\boldsymbol{\alpha}\), and \(\boldsymbol{\omega}\). By the Newton/Euler equations the mass $\boldsymbol{m}$ and the inertia tensor $\boldsymbol{J}$ of the payload can be calculated. To get information about the center of mass $\boldsymbol{c}$ and due to having more unknowns than equations in a single measurement in this case, a least squares approach can be used with multiple data sets to estimate $\boldsymbol{m}$, $\boldsymbol{c}$, and $\boldsymbol{J}$, minimizing the overall error and providing an optimal solution \cite{kurdas2022online, nadeau2022fast}.
    
    Handling \(\phi_{\text{payload}}\) leads to a f/t-measurement reflecting \(\phi_{\text{effective}}\) now. So it clears out \(\phi_{\text{gripper}}\) needs to be isolated to get to the $\boldsymbol{m}$ and $\boldsymbol{J}$ of the \(\phi_{\text{payload}}\).
    
    To isolate the force/torque measurements specific to \(\phi_{\text{gripper}}\) while the robot carries both \(\phi_{\text{gripper}}\) and \(\phi_{\text{payload}}\), estimation is required. Additionally, as the motor sensors register \(\phi_{\text{effective}}\) — the combined effect of both gripper and payload — this estimation is necessary to separate these contributions.
    
    Acceleration, angular acceleration and angular velocity show up as independent features in our motion in relation to our \(\phi_{\text{payload}}\), as long as the manipulation of \(\phi_{\text{effective}}\) does not exceed the maximum motor loads \cite{urrea2018parameter}, (Sec.~\ref{sec:Background}).
    
    In contrast the effort shows up as dependent feature in relation to \(\phi_{\text{payload}}\). The motors efforts increase proportional with increasing \(\phi_{\text{payload}}\), while \(\phi_{\text{gripper}}\) always remains the same~\eqref{eq:I_payload}, (Sec.~\ref{sec:Background}).
    
    To isolate the f/t measurements at the MF resulting from \(\phi_{\text{gripper}}\) from the actual ft measurement at the MF, the sensor data used must be reflected. The transmission function from the motor torques (that are based on the effort sensor data \(\boldsymbol{I}\) measured ~\eqref{eq:tau_motor},~\eqref{eq:dataset_raw}) to the resulting actual f/t measurement at the MF~\eqref{eq:tau_motor} is non-linear due to the complex interactions within the system~\eqref{eq:f_measured}. This non-linearity arises from factors such as the dynamics of joint coupling, friction, and gear reductions, which do not scale linearly with torque. Additionally, the influence of the center of mass and the changing inertia properties of the robot while it moves contribute to this non-linearity~\eqref{eq:robot_dynamics},~\eqref{eq:f_measured}, \cite{liu2019end, blumberg2023estimation}.
    
    Taking the non linearity and the fact we need to get the f/t-measurement resulting of \(\phi_{\text{gripper}}\), while moving \(\phi_{\text{effective}}\), results in two estimation models. 
    
    The effort sensor data shows up as a key feature here because it directly connects $f/t$, $\boldsymbol{a}$, $\boldsymbol{\alpha}$, and $\phi_{\text{effective}}$ (see Eqs.~\eqref{eq:rigidBody} and \eqref{eq:newtonEuler}, and Sec.~\ref{sec:Background}).

     
    Now two GP Regression (Sec.~\ref{subsec:gaussian}) models are trained based on motion data while manipulation with \(\phi_{\text{gripper}}\). The \(\text{GP}_{\text{effort}}\) model's predicted targets are the effort values of each motor, based on the motion with \(\phi_{\text{gripper}}\) \cite{kurdas2022online, urrea2018parameter}. The \(\text{GP}_{\text{wrench}}\) model's targets are the ft values at the measurement frame based on the motion with \(\phi_{\text{gripper}}\) \cite{kurdas2022online, urrea2018parameter}.

    This approach for estimating external forces resulting from a manipulated payload is implemented with a simulated Universal UR5 Robot \cite{ur5_robot} in a Robot Operation System (ROS) \cite{ros_official} Gazebo \cite{gazebo_ros} physics simulation.