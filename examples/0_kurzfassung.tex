% \chapter{Kurzfassung}
    {
     Im Kontext der digitalen Fabrik an der UAS Technikum Wien, wo Menschen und Roboter sich die Aufgaben und den Arbeitsbereich teilen, 
    ist die sichere und effiziente Handhabung von Nutzlasten von entscheidender Bedeutung. In der digitalen Fabrik der UAS werden Nutzlasten derzeit noch ohne Kenntnis 
    ihrer internen Parameter gehandhabt, was zu potenziellen Manipulationsfehlern führen kann, die Menschen Schaden zufügen. Diese Studie beschreibt 
    die Entwicklung einer fortschrittlichen Methode zur Kraft-/Drehmomentabschätzung, um die Fähigkeit eines UR5-Roboters zu verbessern, verschiedene Nutzlastbedingungen zu erkennen und zu handhaben. 
    Diese Fähigkeit gewährleistet die Wahrnehmung des auf einer mobilen Industrieroboterplattform montierten UR5-Roboters, um den sicheren und effizienten Transfer
    von Nutzlasten zwischen verschiedenen Arbeitsbereichen innerhalb der Fabrik zu erleichtern. Die modernsten Methoden zur Kraft-/Drehmomentabschätzung für Industrieroboter nutzen 
    neuronale Netze und Gauß-Prozesse als führende Methoden für genaue Nutzlastabschätzungen. Es wurde ein Gauß-Prozess-Modell entwickelt, um 
    die Kräfte und Drehmomente abzuschätzen, die vom Roboter bei der Ausführung von Trajektorien erzeugt werden. In einem zukünftigen Projekt kann das Bewusstsein für Nutzlasten 
    auf dem UR5-Roboter hinzugefügt werden. Auf diese Weise zielt die Studie darauf ab, die Intelligenz von Robotersystemen in industriellen Umgebungen zu verbessern und den Weg für eine höhere 
    Produktivität und Sicherheit in digitalen Fertigungsumgebungen zu ebnen. Dieses Projekt führte auch zu einer Simulation, die eine Grundlage für die Aufzeichnung der 
    Sensordaten aus dem UR5-Interieur.
    }