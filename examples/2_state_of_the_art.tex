\chapter{State of the Art}

    A focus in the current human-robot collaboration research is external force estimation for collaborative
    industrial robots research \cite{bai2024sensorless, Popov2019, nadeau2022fast, kurdas2022online, su2021deep}. 
    Due to the non-linearity in robotic systems, arising from factors such as the dynamics of joint coupling, friction, and gear reductions, 
    various estimation methods have been established. A distinction is made between Classical Estimation Methods and Machine Learning and Neuronal 
    Network Estimation Methods. Classical estimation methods are techniques based on mathematical models, optimization, and statistical analysis used 
    to estimate unknown parameters or states of a system. These methods typically involve deterministic or probabilistic approaches that are derived 
    from physical models, sensor data, and known system behaviors. They do not rely on data-driven learning algorithms but instead use analytical 
    solutions and optimization frameworks to minimize errors or maximize the accuracy of parameter estimation. Classical estimation methods include 
    Optimization-Based Methods (e.g., Least Squares (LS), Weighted Least Squares (WLS), 
    Iterative Reweighted Least Squares (IRLS) \cite{dong2023dynamic, nadeau2022fast, golluccio2021robot, Xu2022accurate}), 
    Model-Based and Analytical Techniques (e.g. Momentum Observer \cite{kurdas2022online}), and Statistical Techniques (e.g., Kalman Filter, 
    Principal Component Analysis (PCA) \cite{liu2021sensorless,motor_current_estimation, berger2021feature}).
    Machine learning and neural network estimation methods use data-driven algorithms to model and predict unknown system parameters or states~\cite{ren2020learning, su2021deep}. 
    These methods learn from historical data and can adapt to complex, non-linear relationships that may be difficult to model explicitly with classical approaches. 
    The use of neural networks allows for pattern recognition and predictions based on input features, without needing explicit programming for each 
    situation~\cite{zeng2019tossingbot, Kruzic2021}. Machine learning and neural network estimation methods include Neural Network Architectures 
    (e.g., Feedforward Neural Networks~\cite{su2021deep}, Recurrent Neural Networks (RNNs)~\cite{berger2016estimating}, 
    Long Short-Term Memory (LSTM)~\cite{berger2021feature, Kruzic2021}, Convolutional Neural Networks (CNNs)~\cite{Kruzic2021, zeng2019tossingbot}), 
    Advanced Learning Models (e.g., Generative Adversarial Networks (GANs)~\cite{ren2020learning}, Conditional GANs (CGANs)~\cite{ren2020learning}, 
    Least Squares GANs (LSGANs)~\cite{ren2020learning}), Probabilistic and Bayesian Methods 
    (e.g., Gaussian Process Regression (GPR \cite{rasmussen2006gaussian})~\cite{haninger2022model, beckers2017stable,nguyen2008learning, HANINGER2023104431}, 
    Hybrid GPR with Joint Stiffness Models \cite{blumberg2023estimation}, 
    and Experience-Based and Feature Learning (e.g. Experience-Based Torque Estimation \cite{berger2016estimating, berger2015learning}).
    These estimation methods are employed to gather information about various parameters in robotic control and motion. 
    Estimating the dynamic model parameters of robotic systems that are not interacting with their environment focuses on optimizing the robot's trajectory control 
    \cite{lee2020adaptive} and torque control \cite{nguyen2008learning}. Additionally, understanding the inverse kinematics \cite{ren2020learning} and inverse dynamics 
    \cite{leon2022parameter, golluccio2021robot} of the robotic system~\eqref{eq:robot_dynamics} serves as a foundation for external contact force estimation. 
    Here classical estimation methods are present\cite{dong2023dynamic, golluccio2021robot, deng2021dynamic}, as well as machine learning \cite{nguyen2008learning} 
    and neuronal network estimation methods \cite{leon2022parameter,urrea2018parameter}.

    For external contact force estimation, two key scenarios are considered. The first involves estimating contact forces at the robot's links \cite{Popov2019, su2021deep}, either with \cite{Popov2019} or without \cite{su2021deep} determining the location of the force on the link's surface, accounting for multiple contact points with the environment. The second scenario focuses on external contact force estimation at the TCP (Tool Center Point), where only the TCP interacts with the environment. TCP external contact force estimations are crucial for trajectory and control optimization \cite{HANINGER2023104431, haninger2022model}, torque control \cite{motor_current_estimation}, and detecting and minimizing positioning errors \cite{Tan2023, lu2023external}. This involves estimating \(\phi_{\text{effective}}\). Classical TCP contact force estimation \cite{motor_current_estimation}, machine learning methods \cite{HANINGER2023104431, haninger2022model, beckers2017stable, berger2015learning,blumberg2023estimation} and neuronal network estimation models \cite{Tan2023, lu2023external, Kruzic2021, shan2024fine, berger2016estimating, liu2021sensorless, su2021deep} reaching good results in control optimization and position error detection. 
    
    Furthermore, TCP external force estimation is used to gain insights into the internal parameters of the payload. In such cases, the contact force at the TCP is differentiated into \(\phi_{\text{gripper}}\) and \(\phi_{\text{payload}}\)~\eqref{eq:rigidEffective}. Both classical estimation methods \cite{Hu2020, Xu2022accurate, kurdas2022online, nadeau2022fast} and neuronal network estimation methods \cite{berger2021feature} are utilized. 

    Since the dynamic force equation~\eqref{eq:robot_dynamics} defines the internal parameters of the robots rigid body \(\phi_{\text{robot}}\) while motion and shows that \(\tau_{\text{motor}}\), especially \(\mathbf{I}\) is the key feature in this estimation approach~\eqref{eq:tau_motor}~\eqref{eq:tau_I}~\eqref{eq:I_payload}, the Newton/Euler equations~\eqref{eq:newtonEuler} give the conditions to calculate the internal parameters of a rigid body \(\phi_{\text{effective}}\)~\eqref{eq:rigidBody} based on the trajectory parameters \(\mathbf{a}\), \(\boldsymbol{\alpha}\), and \(\boldsymbol{\omega}\). The literature has shown that a parameter identification using the Newton/Euler equations leads to valid information of the the rigid body's \(\phi_{\text{robot}}\) \cite{lee2020adaptive, nguyen2008learning, ren2020learning, leon2022parameter, dong2023dynamic, golluccio2021robot, deng2021dynamic}, \(\phi_{\text{effective}}\) \cite{haninger2022model, HANINGER2023104431, beckers2017stable, berger2015learning,blumberg2023estimation, motor_current_estimation, an2023, lu2023external, Kruzic2021, shan2024fine, berger2016estimating, liu2021sensorless, su2021deep} and \(\phi_{\text{payload}}\) \cite{Hu2020, Xu2022accurate, kurdas2022online, nadeau2022fast} while motion. Also GP Regression models show their ability to estimate various parameters in this context ~\cite{haninger2022model, beckers2017stable,nguyen2008learning, HANINGER2023104431,blumberg2023estimation,berger2015learning}, which is why it is chosen to overcome the non-linearity of in the robot dynamics with respect to the motor sensor data and the resulting f/t-measurements at the MF and to isolate \(\phi_{\text{payload}}\) from \(\phi_{\text{effective}}\).
%%%%%%%%%%%%%%%%%%%%%%%%%%%%%%%%%%%%%%%%%%%%%%%%%%%%%%%%%%%%%%%%%%%%%%%%%%%%%%%%%%%%%%%%%%%%%%%%%%%%%%%%%%%%%%%%%%%%%%%%%%%%%%%%%%%%%%%%%%%%%%%   
    
    \section{One-page condensed SoA summary for Q1}
    \label{sec:Q1_SoA_summary}

    Category Q1 groups classical \textbf{model-based methods for robot and payload dynamics and interaction force estimation}, mostly based on \textbf{linearly parameterised rigid-body dynamics (RBD) and LS/WLS regressors}, sometimes combined with observers and Kalman filters.

Across the papers, \textbf{Least Squares (LS), Weighted LS (WLS) and LS--Newton--Euler (LS--NE) regressors} are the dominant tools for both \textbf{robot dynamic parameter identification (RDPI)} and \textbf{payload dynamic parameter identification (PDPI)}. They are used in joint space, in motor-current space and in sensor frames, and provide \textbf{strong performance for mass, centre of mass (CoM) and joint-torque prediction} when trajectories are sufficiently exciting.

A large subset of works demonstrates that \textbf{neither a nominal CAD-based RBD model nor an FT sensor is strictly necessary}. Q1.1, Q1.5--Q1.8, Q1.13, Q1.15 and Q1.16 build the regressor directly from measured joint states and controller torques, sometimes in \textbf{fully decoupled formulations} or via \textbf{residual-torque decomposition}. They use \textbf{constant-velocity/acceleration S-curve trajectories, Fourier trajectories or repeated sections} to decorrelate parameters and improve conditioning. These approaches typically obtain \textbf{very good mass estimates and acceptable CoM}, with \textbf{inertia remaining the weakest part of the identification}, especially for short trajectories.

Several methods employ \textbf{two-stage pipelines}: static poses for mass and CoM, followed by dynamic trajectories for inertia (e.g.\ Q1.5 and Q1.7). Q1.7 also shows that such schemes scale to \textbf{heavy ($\sim 40\,\mathrm{kg}$) payloads}, and can feed into \textbf{contact force estimation and compensation} with moderate batch times ($\approx 10\,\mathrm{s}$ for contact, $\approx 40\,\mathrm{s}$ for payload).

Where an \textbf{NRB model is available or identified offline}, it is commonly combined with \textbf{observers} for external torque and force estimation. Q1.2 and Q1.3 use LS--NE-based torque prediction together with \textbf{momentum or sliding-mode observers} to estimate external joint torques and EE forces. Q1.4 and Q1.17 combine RBD with \textbf{(adaptive) Kalman filters / disturbance observers} and explicit friction models (Stribeck or NN-based). These approaches can yield \textbf{good EE force estimation and collision detection}, but they are sensitive to model mismatch and friction modelling; Q1.17 reports good behaviour without external forces but large errors (up to $\approx 9\,\mathrm{Nm}$) under contact.
\textbf{Sensorless interaction-force estimation} is addressed in Q1.3, Q1.12 and Q1.17. Q1.12, for example, uses LS--NE identification of $M(q)$, $C(q)$ and $G(q)$, then runs a \textbf{High-Order Finite-Time Observer (HOFFTO)} in joint space, using an FT sensor only as ground truth. This yields \textbf{good joint-torque prediction and acceptable EE force estimates}, but still depends on accurate offline dynamics.
    Finally, Q1.14 shows that combining LS--NE predicted torques with measured joint torques enables \textbf{robust collision detection and localisation of the collided joint} using simple residual thresholds, again assuming a reasonably accurate NRB.

In summary, \textbf{Q1 methods show that classical LS-type identification and observers are mature and effective}:
\begin{itemize}
    \item \textbf{Mass and CoM} can be identified very reliably, even \textbf{without NRB and without FT sensors}.
    \item \textbf{Inertia} is consistently harder and requires \textbf{carefully designed dynamic excitation}, and still tends to be less accurate or weakly validated.
    \item \textbf{Torque prediction, contact detection and simple EE force estimation} are already at a high level with these methods, but they \textbf{rely on good friction modelling and reasonably accurate dynamics}.
\end{itemize}
